\section{Introduction and Goals}\label{6_introduction}
The SLS familiarize beginners with the slackline and provide an appropriate learning structure to teach them the basics of standing and walking on a slackline.
A conducted study measures and evaluates the learning progress of beginners on a slackline with the SLS.
Further, it should show whether such a system motivates a beginner that is interested into learning slacklining.
Moreover, if the participants are interested in using such a learning system and in which field in the real world it could be applied.
%useful

Like stated in Section \todo{section} a personal human trainer is the common way of learning slacklining.
Therefore, the SLS is compared against this method to show if it can compete with it.
It is a \textit{think aloud} study, which means the participant shares her thoughts, informs the study leader about the way she thinks when making an action, and where she has problems.

Several research goals arise that are analysed and discussed, which are the following:

\begin{itemize}
\item The SLS supports the participant in learning slacklining and during the execution
\item The SLS has some statistically relevant influence to the learning progress of beginners on a slackline
\item It motivates the user in learning new skills and techniques for slacklining
\item The system is usable for learning new skills in the field of sport slacklining
\item The SLS can compete with a personal human trainer, as common training method for teaching beginners on a slackline
\item The SLS shows similar effects of learning progress of beginner so a slackline as a human trainer
\item The structure of the exercises provided by the system are challenging, ascending in its difficulty, and show positive effects in the learning progress
\item The real difficulty of the exercises match the subjective difficulty perception of the participants
\end{itemize}
