\section{Introduction and Research Questions}\label{6_introduction}
The SLS familiarizes beginners with the slackline and provides an appropriate learning structure to teach them the basics of standing and walking on a slackline.
The conducted study measures and evaluates the learning progress of beginners on a slackline with the SLS and shows whether it motivates a beginner who is interested in learning slacklining.
Furthermore, it points out, if the participants are interested in using such a learning system and where it could be applied in the real world.
%useful

A personal human trainer is the common way of learning slacklining because she can provide immediate feedback and hints to the trainee.
Therefore, the SLS is compared against this method to demonstrate if it can show similar results and compete with it.
%It is a \textit{think aloud} study, which means the participant shares her thoughts, informs the study leader about the way she thinks when making an action, and where she has problems.
The participant shares also her thoughts, informs the study leader about the way she thinks when making an action, and where problems exist from her point of view.

Several research questions arise that are stated in the following and will be analysed and discussed in the study discussion:

\begin{itemize}
\item Is the system usable for learning new skills in the field of sport slacklining?
\item Can it compete with a personal human trainer, as common training method for teaching beginners on a slackline?
\item Has the SLS statistical relevant influence to the learning progress of beginners on a slackline?
\item Does the SLS support the participant in learning slacklining and during the execution?
\item Is the participant motivated in learning new skills and techniques for slacklining?
\item Is the structure of the exercises provided by the system perceived challenging, ascending in its difficulty, and shows positive effects in the learning progress?
\item Can the real difficulty of the exercises match the subjective difficulty perception of the participants?
\end{itemize}
