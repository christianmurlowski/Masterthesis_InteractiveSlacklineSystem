\section{Method}
\begin{comment}
-- briefing
-- methods
-- pre-/post measurement
-- execution
-- interview
\end{comment}
\subsection{Briefing \& Data Collection}
At first the participant was welcomed and then briefed about the idea as well as what she can expect from the study. Further she was introduced about the method in which she participates. 
After that she had to fill out a questionnaire for collecting demographic data and her prior experience with slacklining. The ISG had to answer one more question about the prior experience with interactive devices like Kinect, Wii, PlayStation Move, and so on.
%Therefore one can see if a person, which tends to have a better experience with the system during the study, relies on her prior experience with such devices. 
The pyhsical activity level was determined with the~\textit{Physical Activity, Exercise and Sport Questionnaire (Bewegungs- und Sportaktivität Fragebogen - BSA-F)} by Fuchs et. al~\cite{Fuchs2015-bsa}. The preference of any body side was determined with a \textit{Lateral Preference Inventory Questionnaire} by Coren~\cite{Coren1993-lp}. At last she had to confirm a form consent.

\subsection{\todo{Procedure/Slackline training}}
\subsubsection{Interactive Slackline System Group}
The participant had to stay on a marked position in front of the slackline. Like seen in chapter~\ref{5_systemIntegration} the participant has to interact at her own with the system. It teaches the user how to interact with it and guides her through predefined exercises. Therefore it explains on her own how to execute the exercises, how many repetitions to make, and in which time to accomplish each repetition. It provides real-time feedback about the performance. The participant could ask to skip the exercise if it were too difficult to accomplish or not recognisable by the system. The experiment leader had no influence about questions regarding the exercise execution to ensure the autonomy of the user with the system.
\subsubsection{Human Trainer Group}

\subsection{Measures}
The general balance ability of the participant was measured before training by how long she can stand on the ground and a towel. Comparing the results of the training method were conducted by standing on a slackline before and after the training. All measurements involves the left, right, and both feet and were counted in seconds. 

Furthermore, the participant had to made as many steps as she can on the slackline with the left and right foot as starting point. \todo{Hereby the steps were counted as comparison parameter}. Additionally the distance on the slackline was measured by dividing the device into five areas, each with a distance of 60 cm. Three attempts per side and method were executed and \todo{the best taken / the average calculated} to compare the results.

\todo{Dependent variables --> }
\begin{itemize}
\item Time stood on line with left, right, both feet
\item As many steps as possible on the line ( + distance with markers on the line) left, right feet
\item Accomplished exercises
\end{itemize}

\todo{Independent variables --> training method}
\begin{itemize}
\item Interactive Slackline System
\item Human Trainer Group
\end{itemize}

\todo{Confounding variable}
\begin{itemize}
\item Experience with general balance training
\item Experience with slacklining
\item General physical activity
\end{itemize}

%A measurement of the participants' current balance performance was conducted before and after the training to compare the training results and learning progress.  This involves the measurement of how long the participant can stand on the slackline in seconds with her left, right, and with both feet. Further, how many steps were she able to walk on the slackline with the left or right foot for getting up the line. Three attempts per side and method were executed and \todo{the best taken / the average calculated} to compare the results.

\subsection{\todo{Interview / Execution Questions}}
- questions during execution for each exercise how difficult

-- subjective thought of participant vs. her real performance (assumption/self accessed vs. logged data/parameter)

-- If exercise set difficulty of each level is really ascending

- interview questions
%for reviewing and validating performance parameters as well as to detect system failures.


