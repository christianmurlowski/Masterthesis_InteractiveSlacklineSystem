\section{Structure}

\subsection{Briefing \& Data Collection}
At first the participant was welcomed and then briefed about the idea as well as what she can expect from the study. Further she was introduced about the method in which she participates. 
After that she had to fill out a questionnaire for collecting demographic data and her prior experience with slacklining. The ISG had to answer one more question about the prior experience with interactive devices like Kinect, Wii, PlayStation Move, and so on.
%Therefore one can see if a person, which tends to have a better experience with the system during the study, relies on her prior experience with such devices. 
The pyhsical activity level was determined with the~\textit{Physical Activity, Exercise and Sport Questionnaire (Bewegungs- und Sportaktivität Fragebogen - BSA-F)} by Fuchs et. al~\cite{Fuchs2015-bsa}. The preference of any body side was determined with a \textit{Lateral Preference Inventory Questionnaire} by Coren~\cite{Coren1993-lp}. At last she had to agree for participating the study and confirm to be recorded via audio and video for later analysis.

\subsection{\todo{Procedure/Experiment}}

\subsection{Measures}
The general balance ability of the participant was measured before training by how long she can stand on the ground and a towel. Comparing the results of the training method were conducted by standing on a slackline before and after the training. All measurements involves the left, right, and both feet and were counted in seconds. 

Furthermore, the participant had to walk as far as she can on the slackline with the left and right foot as starting point. Hereby the steps were counted as comparison parameter. Three attempts per side and method were executed and \todo{the best taken / the average calculated} to compare the results.

Lastly, the accomplished exercises by the participants of each group were counted. Since the training time was restricted to 40 minutes not all trainees were able to finish all exercises.

\todo{dependent variables --> }
\begin{itemize}
\item Time stood on line with left, right, both feet
\item Steps walked on line with left, right feet
\item Accomplished exercises
\end{itemize}

\todo{independent variables --> training method}
\begin{itemize}
\item Interactive Slackline System
\item Human Trainer Group
\end{itemize}

\todo{Confounding variable}
\begin{itemize}
\item Experience with general balance training
\item Experience with slacklining
\item General physical activity
\end{itemize}

%A measurement of the participants' current balance performance was conducted before and after the training to compare the training results and learning progress.  This involves the measurement of how long the participant can stand on the slackline in seconds with her left, right, and with both feet. Further, how many steps were she able to walk on the slackline with the left or right foot for getting up the line. Three attempts per side and method were executed and \todo{the best taken / the average calculated} to compare the results.

\subsection{\todo{Interview / Execution Questions}}
- questions during execution for each exercise how difficult

- interview questions
%for reviewing and validating performance parameters as well as to detect system failures.


