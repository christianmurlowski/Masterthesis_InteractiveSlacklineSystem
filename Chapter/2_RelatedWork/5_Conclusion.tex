\section{Conclusion}
With the stated related work a foundation is given to build a slacklining assistance system. For teaching beginners on a slackline it is important get familiar with it. The assistance system should provide the given exercises and tips for beginners which build a foundation for further training.
%like focusing the eye and turning the hands over the shoulder, 
Several application scenarios show that slacklining can replace balance training in rehabilitation environment, as prevention system, in school sport or as an home assistance. This can be combined with interactive technology, which helps patients to fulfil their exercises and provide the medical stuff with sufficient analysis data.

As interaction device the Microsoft Kinect v2 seems like the best choice out of the available technologies. It provides sufficient useful and accurate data analysis, if no in-depth analysis is needed. More advantages are the low cost, short setup time and the freedom of the movements for the user. Several studies indicate also that the Kinect can be embedded in balance training scenarios and increases the training efficacy while motivate patients.

A problem that occurs with more complexity in the exercises is the raising cognitive load. The system should therefore provide appropriate feedback and be aware of the cognitive load of the slacker. Motivating the slacker for further exercise execution can be done with a well defined interaction mechanism, an enjoyable but challenging virtual training environment, and an user friendly interface. This can be realised especially with the help of human interface guidelines provided by Microsoft, which include several design tips for developing a Kinect application.

With the help of this foundation a concept for the slacklining assistance system has been created, which can be seen in the next chapter.