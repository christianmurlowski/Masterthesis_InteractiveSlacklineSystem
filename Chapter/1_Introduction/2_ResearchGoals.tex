\section{Research Goals}
The research goals of this thesis are divided into four parts. 
First, the exploration of slackline training and the design of interactive tracking technologies.
%interactive systems within the context of balance training, and the comparison of different interactive technologies.
Second, the conceptual design of the actual interactive slackline learning system.
Third, implementation details of the system from the view point of an developer.
Lastly the investigation of the SLS with slackline beginners in a user study to show if a positive learning effect can be achieved and if it can compete against a common learning method.
These parts are further described in the following.
%on the concept, implementation, and investigation of such a system, which are further described in the following:

\subsubsection{Exploration of slackline training and designing appealing interactive systems}
At first the usage of slacklining and its training effects the human body will be shown by several works
Further, several interactive tracking devices and applications in balance training will be compared.
It is also important how to provide appropriate feedback and designing an appealing application in the context of balance sport training.
This helps to get an idea on how to design the application for the SLS and which factors should be considered to motivate the trainee as well as providing an adequate user experience.
Several learning methods exist to train beginners on a slackline.
Mainly the investigations of Thomann ~\cite{Thomann2013-aa} and Kroiß~\cite{Kroiss2007-ab} are used to provide an appropriate training method and exercises. 

\subsubsection{Conceptual design of a prototypical interactive slackline learning system}
The conceptual design will be described on the basis of the related work and the slackline exercises.
%It is independent of the used tracking technology.
The used tracking technology was not taken into account for the concept.
Hence, it provides the possibility to adapt it for other tracking systems.
It contains basic interaction design principles, specific basics for the system regarding slacklining, structuring of the exercises, and the feedback system.

\subsubsection{Implementation for the usage in a user study}
The system implementation process consists of the technical development part based on the conceptual elaboration for the study afterwards.
It describes in general hardware components, apparatus, data management, movement recognition, and user interface.
Special attention will be payed at the positioning of the slackline to the Kinect as tracking device and the coherence of the Kinect with the development platform Unity are part of this.

\subsubsection{Investigation of the slackline learning system}
A user study will be conducted to show the usefulness of the system, its strengths and weaknesses, and whether the SLS shows positive effects on the learning progress of beginners on a slackline. To explore if such a system can compete with a common learning method it will be confronted with personal trainer.
Therefore, the study is divided in two groups in which one trains with the SLS and the other with a human personal trainer.
The performance parameters of single leg stance on the slackline, how many steps they can walk with each leg, and the distance the walk on the line will be measured before and after the training.
Therefore, the learning progress of the users' balance ability can be shown.
Lastly, an interview should reveal the participants experience with the training method after the training.
%and the performance of the exercise execution for each condition will also be analysed.
%This includes the single leg stance on the slackline with each foot and to walk on a slackline with as many steps as possible.

%This thesis investigates if an interactive slackline learning system with real-time, further called SLS, enhances the learning progress of beginners on a slackline.
%Furthermore, if it can compete against a personal human trainer as common learning condition  and resolve the stated problems.
%Lastly if such a system motivates the user and delivers an appropriate user experience.
%and would be a useful home training system from the perspective of the user.


\begin{comment}
- Investigate related system

- Requirement analysis

- Conceptual design of an interactive feedback system for slacklining
-- Provide supportive feedback

- User interface design

- Integration

- Investigation of the system
\end{comment}
%\subsection{Hypothesis}
\begin{comment}

- Show if an interactive real time feedback system is usable for this kind of sport

- If the learning progress is comparable with other training methods like human trainer

- If such a system motivates user for slackline exercises
\end{comment}