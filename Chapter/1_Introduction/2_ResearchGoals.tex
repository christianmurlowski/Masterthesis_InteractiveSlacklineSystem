\section{Research Goals}
This thesis investigates if an interactive slackline learning system with real-time can compete with a personal human trainer and resolve the stated problems. 
Furthermore, if such a system would be a useful home training system from the perspective of the user.
The main focus lies on the concept, implementation, and investigation of such a system, which are further described in the following:

\subsubsection{Exploration of training methods in slacklining and interactive systems in exergaming}
- what slackline methods exists, how to learn it properly

- How do other works designed their systems

\subsubsection{Conceptual design of a prototypical interactive slackline learning system}
- Concept regardless of tracking technology

- Design of an appealing and motivating user interface

- which feedback parameters are useful and should be integrated

\subsubsection{Implementation for the usage in a user study}
- implementation process

\subsubsection{Investigation of the slackline learning system}
- user study
- human trainer vs system
- parameters to measure
- analyse
- usage of the system

\begin{comment}
- Investigate related system

- Requirement analysis

- Conceptual design of an interactive feedback system for slacklining
-- Provide supportive feedback

- User interface design

- Integration

- Investigation of the system
\end{comment}
\subsection{Hypothesis}
\begin{comment}

- Show if an interactive real time feedback system is usable for this kind of sport

- If the learning progress is comparable with other training methods like human trainer

- If such a system motivates user for slackline exercises
\end{comment}