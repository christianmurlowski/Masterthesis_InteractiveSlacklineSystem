\section{Research Goals}
This thesis investigates if an interactive slackline learning system with real-time, further called SLS, can compete with a personal human trainer and resolve the stated problems. 
Furthermore, if such a system would be a useful home training system from the perspective of the user.
The main focus lies on the concept, implementation, and investigation of such a system, which are further described in the following:

\subsubsection{Exploration of designing interactive systems in exergaming and training methods in slacklining}
At first several related works regarding interactive systems with an exergaming approach will be elaborated. 
This helps to get an idea on how to design the application for the SLS and which factors should be considered to motivate the trainee as well as providing an adequate user experience.
Several learning methods exist to train beginners on a slackline.
Mainly the investigations of Thomann ~\cite{Thomann2013-aa} and Kroiß~\cite{Kroiss2007-ab} are used to provide an appropriate training method and exercises. 

\subsubsection{Conceptual design of a prototypical interactive slackline learning system}
The conceptual design is described regardless of the used tracking technology.
Hence, it can be adapted for arbitrary tracking systems.
The concept it contains basic interaction design principles, specific basics for the system regarding slacklining, structuring of the exercises, and the feedback system.

\subsubsection{Implementation for the usage in a user study}
The system implementation process consists of the technical development part based on the conceptual elaboration for the study afterwards. 
The hardware components, apparatus, data management, movement recognition, and user interface are part of this.

\subsubsection{Investigation of the slackline learning system}
The thesis investigates whether the SLS can show positive effects on the learning progress of beginners on a slackline and if it is a  useful alternative to common learning methods.
Therefore a user study is conducted with two groups.
One group trains with the SLS and the other group will be trained with a human personal trainer.
The performance parameters of single leg stance on the slackline and how many steps they can walk on the line will be measured before and after the training and then compared.
Therefore, the learning progress of the users' balance ability can be shown.
%and the performance of the exercise execution for each condition will also be analysed.
%This includes the single leg stance on the slackline with each foot and to walk on a slackline with as many steps as possible.


\begin{comment}
- Investigate related system

- Requirement analysis

- Conceptual design of an interactive feedback system for slacklining
-- Provide supportive feedback

- User interface design

- Integration

- Investigation of the system
\end{comment}
\subsection{Hypothesis}
\begin{comment}

- Show if an interactive real time feedback system is usable for this kind of sport

- If the learning progress is comparable with other training methods like human trainer

- If such a system motivates user for slackline exercises
\end{comment}