\section{Motivation}
%- Interactive real time feedback system in other sports
With the realease of the Wii in year \todo{year} a new area of game interaction has been released. It was possible to interact an application with a controller that adapts the users hand. The hand is here part of the controlling device. Hereby new interaction possibilities resulted out of this. For example in a baseball game the controller represents the baseball bat. The user had to swing the controller like in the real world (Figure \todo{FIGURE}). Specific characteristics of games were hereby more realistic, since the controller could be adapted as main interaction with the virtual gaming device. 

\todo{Figure baseball bat}

Other manufacturers conquered with their own system, like the PlayStation Move or Microsoft Kinect. The PlayStation Move implemented a similar approach like the Nintendo Wii, with a controller that serves as pointing device. Since the interaction of these devices are restricted to a controller the Kinect followed another approach. It was possible to interact without any controlling device but just with the users hands. This again resulted in even more possibilities. Fitness applications found its application in the world of interactive technology devices with the Kinect. Especially with exercises where the user needs both hands or should not be disturbed by any other device, like push ups or crunches (Figure \todo{FIGURE}).

\todo{Figure fitness applications}

The combination of such exercise applications in the context of a gamified environment and approach originated in a new gaming area with the name \textit{exergames}.

\textbf{-> Interaktive Systeme auf dem Markt}

-- Größtenteils genutzt für gaming

-- aber auch exergames großes potential

-- Home Training Assistent

\textbf{-> Einführung von exergames hiermit}

-- What are Exergames in general

-- Exergames positive impact on skill acquisition --> more exergames released in past decades?
---> Combining interactive technology in the context of learning or training skill for a sport activity within an application in a gaming approach is called

-- Several sports implemented

-- für Rehabilitation, krankengymnastik genutzt

-- Sportarten trainieren

-- Balance Sportarten/Activities

\textbf{-> Einführung von Slackline hiermit}

-- Kurze Slackline description

-- Slackline als sportart einführen

-- Normales lernprozedere erklären und assisstenzsystem erläutern / Mostly skill acquisition is done with a professional who already knows how to act on the line and where particular attention must be paid

-- Keine vergleichbare Interaktiven Anwendung für das Slacklinen / No comparable work relating to slacklining

-- In contrast to such a human personal trainer this thesis will elaborate an interactive slackline learning system with real-time feedback, called further SLS

-- Support slackline beginners with such a system
%\end{comment}