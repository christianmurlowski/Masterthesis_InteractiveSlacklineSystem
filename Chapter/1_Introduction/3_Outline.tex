\section{Outline}
The structure of the thesis outlined in the following.
As a groundwork the \textit{\nameref{2_relatedWork}} chapter involves sections of \textit{\nameref{2_2_slacklineTraining}}, a comparison of interactive tracking devices in \textit{\nameref{2_3_interactiveTechnology}}, \textit{\nameref{2_4_methods}}, as well as \textit{\nameref{2_4_UIDesign}}. Chapter \textit{\nameref{3_slacklining}} gives an overview about the sport and how to learn slacklining appropriate with section \textit{\nameref{3_3_learningTechniques}}.
The conceptual design is described in chapter \textit{\nameref{4_concept}} with general interaction design principles, interaction with the system, and how to build an appropriate structure related to the slackline exercises elaborated in the previous chapter.
Details of the system development are further described in chapter \textit{\nameref{5_systemIntegration}}, which is divided into the sections \textit{\nameref{5_1_systemSetup}}, \textit{\nameref{5_2_dataModel}}, \textit{\nameref{5_3_movementRecognition}}, and \textit{\nameref{5_4_software}}.
The study structure, methods, results, and discussion can be found in chapter \textit{\nameref{6_study}}. Lastly, chapter \textit{\nameref{7_conclusion}} comprises the findings of the thesis and gives approaches and recommendations for future work.

\begin{comment}
The thesis is structured as follows:
As a groundwork the \textbf{\nameref{2_relatedWork}} chapter involves basics of \textbf{\nameref{2_2_slacklineTraining}}, a number of possible interactive tracking devices that will be compared in the section \textbf{\nameref{2_3_interactiveTechnology}}, and \textbf{\nameref{2_4_methods}}.

- \todo{Further chapters}

A List of \textbf{\nameref{tablesRef}}, \textbf{\nameref{listFiguresRef}}, \textbf{\nameref{listAbbreviationsRef}} and the \textbf{\nameref{bibliographyRef}}  can be found at the end of the thesis.

\end{comment}