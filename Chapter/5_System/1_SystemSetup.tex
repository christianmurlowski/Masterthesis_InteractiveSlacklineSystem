\section{Hardware}\label{5_1_systemSetup}
\subsection{Components}\label{5_1_hardwareComponents}
%- System architecture of system --> Unity3D, Kinect SDK, Kinectstudio, VGB --> kinect sdk free to use since version X
In the following several hardware components of the system architecture will be described. Each component is necessary for the functionality of the SLS and for the study afterwards. An overview can be seen in figure~\ref{fig:5_3_systemArchitecture}.
% Kinect, Beamer, Screen, Slackline --> Alpidex High Performance

Since the focus of this thesis lies mainly on beginners a short mobile slackline, namely \textit{alpidex POWER-WAVE 2.0}\footnote{\url{http://www.alpidex.com/fitness/slacklines/slackline-gestell-in-2-laengen-power-wave-2-0-inklusive-slackline/a-10288/}} was used.
It provides the needed mobility and independency due to its comparatively short length of three meters.
Hence it is possible to set it up indoors as well as move it in different positions with a minimum of effort.
The included slackline is tensed around brackets at both ends of the device.
%The middle rail is divided into two parts and needs to be put together.
It is placed in front of the \textit{Microsoft Kinect v2}, which is used as tracking device, like discussed in section~\textit{\nameref{trackingTechnologie}}. The Kinect itself is attached on a \todo{modell} tripod with a height of about \todo{height}.
A \textit{STEAM® MACHINE PC by ZOTAC}~\footnote{Technical details of the STEAM® MACHINE: Intel Core i5-6400T @ 2.2 GHz, NVIDIA GeForce® GTX 960, 8 GB RAM} served as development device that fulfilled the recommended specs of the Kinect: \textit{Windows 8, 4 GB Memory, Physical dual-core processor with 3.1 GHz or faster, USB 3.0 Gen-2 controller, Graphics card supporting DirectX 11}. As visual output device a projector \todo{modell} with a resolution of 1920x1080 was attached on a traverse system.
The interface was visualized on a projector screen with a size of \todo{2x3 m} to give the user a more immersive feeling.

\todo{vlt eher bild vom gesamten setup hier rein?}
\begin{figure}[htb]
	\centering
	\begin{minipage}[t]{1\linewidth}
		\centering
		\includegraphics[width=1\linewidth]{Pictures/5_3_systemArchitecture}
		\caption{System overview}
		\label{fig:5_3_systemArchitecture}
	\end{minipage}
\end{figure}

\begin{comment}
\begin{figure}[htb]
	\centering
	\begin{minipage}[t]{1\linewidth}
		\centering
		\includegraphics[width=0.44\linewidth]{Pictures/3_2_mobileSlackline}
		\caption{Mobile slackline \textit{alpidex POWER-WAVE 2.0}~\cite{alpidex2017-ms}}
		\label{fig:3_2_mobileSlackline}
	\end{minipage}
\end{figure}
\end{comment}

\subsection{Kinect and Slackline positioning}\label{5_1_technicalFeasibility}
Tracking a person on a slackline with the Kinect is very different from a common situation.
The combination of the slackline range, vibration of the line itself, and unpredictable movements of the user and her balancing actions could lead to imprecise and inaccurate tracking data.
%disturb the tracking performance.
Furthermore, there exists no comparable work about how to track user appropriate on a slackine with the Kinect.

The major point is to compare different slackline positions (Frontal: 0 Degrees, Diagonal: 45 Degrees, Orthogonal: 90 Degrees) regarding multiple angles and heights of the Kinect on a tripod or traverse system (80 cm, 160 cm, 240 cm).
The testing person was recorded via \textit{KinectStudio}~\footnote{\url{https://developer.microsoft.com/de-de/windows/kinect/tools}}, a tool for recording clips out of the streaming data of the Kinect.
This scenario should clarify how good a person can be tracked on a slackline.
Moreover, which is the best combination of the slackline and Kinect positioning to track user on an entire slackline as well as for study purposes of this thesis.

\subsubsection{Limitations of the Kinect} 
A considerable role plays the angle and tracking range of the Kinects' depth sensor regarding the length of the slackline. Its angle of vision covers in horizontal 70 degrees and in vertical 60 degrees (Figure~\ref{fig:5_1_1_visionAngle}). Since the slackline is about 30 cm off the ground it could lead to tracking problems and result in cropped body parts depending on the users' height. The total tracking range of the sensor covers a range between 0.5 and 4.5 meters, whereas the sweet spot area lies between 1 up to 4 meters (Figure~\ref{fig:5_1_1_trackingRange})~\cite{MicrosoftHIG2014-mh}. The mobile slackline device used in this thesis has a length of three meters and therefore fits entirely within the sweet spot.
\begin{figure}[htb]
	\centering
	\begin{minipage}[t]{0.44\linewidth}
		\centering
		\includegraphics[width=1\linewidth]{Pictures/5_1_1_visionAngle}
		\subcaption{Angle of vision}
		\label{fig:5_1_1_visionAngle}
	\end{minipage}
	\hfill
	\begin{minipage}[t]{0.44\linewidth}
		\centering
		\includegraphics[width=1\linewidth]{Pictures/5_1_1_trackingRange}
		\subcaption{Kinect tracking range}
		\label{fig:5_1_1_trackingRange}
	\end{minipage}
	\caption{Sensor constraints of the Kinect v2~\cite{MicrosoftHIG2014-mh}}
	\label{fig:5_1_1_sensorConstraints}
\end{figure}

%\subsubsection{Testing scenario}

%The test took place in the laboratory of the research group in the \textit{german reasearch center for artificial intelligence}~\footnote{\url{https://www.dfki.de/web/kontakt/dfki-saarbruecken}}. A big advantage of this is the large space to place the slackline easily in different variations. 
%The slackline is placed in three positions to the Kinect: frontal (0 Degree), diagonal (45 Degree) and sideways (90 Degree) \textbf{\todo{(Figure X - 1)}}. Each of this positions is tested regarding three different height levels of the Kinect: 80 cm, 160 cm, and 240 cm. Therefore it was attached on a tripod or traverse system (\textbf{\todo{Figure X - 2}}). At the end nine different combinations are covered to track a user on a slackline, which gives a general overview of the Kinect height to the slackline. The testing person was recorded via \textit{KinectStudio}~\footnote{\url{https://developer.microsoft.com/de-de/windows/kinect/tools}}, a tool for recording clips out of the streaming data of the Kinect.
%In the following the results discuss the feasibility and appropriate tracking positions.

\subsubsection{Best positioning for study purposes \todo{kürzen? tabelle unnötig?}}

With a slackline positioned orthogonal (90 Degrees) to the Kinect, no interference regarding the limits of the tracking range can happen. This is because the whole body is in a constant line within the tracking area. However, permanent overlapping of body parts resulted in problems to detect body joints with an appropriate accuracy and precision (\textbf{\todo{Figure X}}).

When placing the slackline diagonal (45 Degrees) to the Kinect several body parts will not overlap because the body is more visible to the sensor.
%the frontal and end point of the slackline now differ in the vertical axis, which is unproblematic  \textbf{\todo{(Table X and Figure X)}}. 
%This could even result in a better trackability in matter of the depth field range, since the distance in the front shrinked and is therefore closer to the Kinect depth view. 
Tracking failure still because the joints of the arms and legs interfere with other body joints, which makes it also unusable \textbf{\todo{(Figure X)}}. 
%This results in a relatively bad skeletal tracking and depending on the executed exercise it can lead to detection problems.

Positioning the slackline frontal to the Kinect, its view can see the full body and track joints without any problems.
Problems with tracking occurred at the starting position of the slackline, because the slackline uses the entire tracking range. Hence the user stands at the outermost limit of this range ~\textbf{\todo{(Figure X)}}.
%Three main height levels were used to show the main differences of the tracking behaviour from the Kinect. It is mounted on a tripod and covers the heights of 0.80, 1.60 and 2.40 meters off the ground. 

Beginning with a height of 2.40 meters the Kinect has a very steep view angle.
%to track the slackers' body on the full range of the slackline. 
As a result of that the tracking range shifts more into the front and shrinks \textbf{\todo{Figure X}}. With this the starting position cannot be arranged for appropriate usage. 
%If beginning at the starting position on the slackline the slacker reaches hereby immediately the end of the tracking area. 
Also the further she walks towards the Kinect the more joints will overlap .

With a height between 1.60 and 0.80 meters the Kinect has a flatter view angle, which results in a more homogeneous tracking range. The body is fully visible in the entire tracking range. (\textbf{\todo{Figure X}}).
%Problems can occur with positioning the Kinect on a lower height. It can lead to cropped body parts like the head or arms at the very end of the slackline.
If the height of the Kinect is about 1.60 m the slackline must be positioned further away from the Kinect to prevent cropped body parts like the head or arms at the very end of the line.

\todo{\textbf{table}}
%The tracking and view is more homogeneous and the angle flatter, which results in the possibility to use the full depth range. With a higher attachment the angle will be too steep, which results in less depth range, as well as more occlusions of body parts can occur.
\rowcolors{2}{tablerowgray}{tablerowgray}
\begin{table}[h!]
\centering
%\arrayrulecolor{white}
\renewcommand{\arraystretch}{1}
\begin{tabular}{ | c | c | c | c | c | c | c | c | }
\hline
\rowcolor{tableheadergray} & \multicolumn{7}{ c| }{\textbf{Slackline Positioning (m)}} \\ 
\rowcolor{tableheadergray} & \multicolumn{2}{ c| }{\textbf{Frontal}} & \multicolumn{2}{ c }{\textbf{Diagonal}} & \multicolumn{3}{ |c| }{\textbf{Sideways}}\\
\rowcolor{tableheadergray} \multirow{-3}{*}{\textbf{Kinect Height (m)}} & \textbf{Front} & \textbf{Back} & \textbf{Front} & \textbf{Back} & \textbf{Front} & \textbf{Back} & \textbf{Mid} \\
\hline
2,40 & 1.30 & 4.30 & 1.90 & 3.80 & 3.00 & 3.00 & 2.70 \\
\hline
1.60 & 1.70 & 4.70 & 2.10 & 4.00 & 3.00 & 3.00 & 2.70 \\
\hline
0.80 & 1.30 & 4.30 & 1.90 & 3.80 & 2.60 & 2.60 & 2.10 \\
\hline
\rowcolor{green} 0.80 - 1.20 & 0.00 & 4.00 & - & - & - & - & - \\
\hline
\end{tabular}
\caption{Demographic data and physical activity table}
\label{table:1}
\end{table}

% But since beginner will use the entire slackline range it can be neglected.
The best combination resulted placing the slackline frontal and having a Kinect height of 0.80 up to 1.20 meters. Hereby, the Kinect can track the entire body with nearly no joint overlap. Since the focus of the study in this thesis lies mainly on beginners, the starting position of the slackline is important and must not lie at the outermost tracking range. Hence, it makes more sense to move the slackline very close to the Kinect, so that it fits well within the sweet spot area.
%a higher tracking confidence is possible.
%plays an important role and 
%because the very end of the line is not necessary for the study.
%a little bit of slackline is cropped out of the view but 
%A higher tracking confidence is possible at the starting position, which is more important in this case \textbf{\todo{(Figure X)}}.
%This results in a relatively bad skeletal tracking and depending on the executed exercise it can lead to detection problems.