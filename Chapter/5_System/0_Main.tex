\chapter{System integration}\label{5_systemIntegration}
This chapter shows the implementation of the interactive learning system with real-time feedback regarding the conceptual elaboration. Like seen in subsection ~\textit{\nameref{trackingTechnologie}} the Microsoft Kinect v2 will be used as tracking technology. Hence before going into the implementation detail section~\textit{\nameref{5_3_technicalFeasibility}} clarifies the feasibility and the performance of tracking persons on a slackline with the Kinect. Further section \textit{\nameref{5_2_gestureConstruction}} describes the recording and training of predefined gestures for the exercises in the system. After this section \textit{\nameref{5_3_systemArchitecture}} describes the general system architecture which involves the interplay of the Kinect SDK with Unity3D as game engine. Further it involves more specific how the data management, engagement, interaction components, gesture integration, and real-time feedback are implemented. Finally section \textit{\nameref{5_4_userInterface}} covers the design process of the application inlcuding scribbles, mockups and the final integration.

\section{Technical feasibility}\label{5_1_technicalFeasibility}
Tracking a person on a slackline is very different from a regular situation. The interplay between the range of the slackline, the movement of the line itself, unpredictable movements of the user, and his balancing actions could possibly disturb the tracking ability. This can then lead to imprecise and inaccurate tracking data. Furthermore there exists no comparable data about how to track user properly on a slackine with the Kinect.

The major point is to compare different slackline positionings and rotations regarding multiple angles and heights of the Kinect. This will then clarify how good a person can be tracked on a slackline and which is the best combination of the slackline and Kinect positioning.

\subsection{Constraints of the Kinect} 
A mentionable role plays the angle and detection range of the Kinects depth sensor regarding the length of the slackline. The sensors angle of vision is in horizontal 70 degrees and in vertical 60 degrees (Figure~\ref{fig:5_1_1_visionAngle}). Hence tracking the users height could result into problems becuause the slackline is about 30 cm off the ground, which in combination with a person on a slackline results in a higher body positioning of the person. The sensors range limits lies between 0.5 and 4.5 meters whereas the sweet spot area is between 1 up to 4 meters~\cite{MicrosoftHIG2014-mh} (Figure~\ref{fig:5_1_1_trackingRange}). Since the mobile slackline device has a length of three meters, it would fit entirely into the sweet spot tracking range. To track user for further training on a longer slackline, the depth range is not sufficient. This could be solved by using more than one Kinect device to have a larger the range. 
\begin{figure}[htb]
	\centering
	\begin{minipage}[t]{0.44\linewidth}
		\centering
		\includegraphics[width=1\linewidth]{Pictures/5_1_1_visionAngle}
		\subcaption{Angle of vision}
		\label{fig:5_1_1_visionAngle}
	\end{minipage}
	\hfill
	\begin{minipage}[t]{0.44\linewidth}
		\centering
		\includegraphics[width=1\linewidth]{Pictures/5_1_1_trackingRange}
		\subcaption{Kinect tracking range}
		\label{fig:5_1_1_trackingRange}
	\end{minipage}
	\caption{Sensor constraints of the Kinect v2~\cite{MicrosoftHIG2014-mh}}
	\label{fig:5_1_1_sensorConstraints}
\end{figure}

\subsection{Testing scenario}

%The test took place in the laboratory of the research group in the \textit{german reasearch center for artificial intelligence}~\footnote{\url{https://www.dfki.de/web/kontakt/dfki-saarbruecken}}. A big advantage of this is the large space to place the slackline easily in different variations. 
The slackline is placed in three positions to the Kinect: frontal (0 Degree), diagonal (45 Degree) and sideways (90 Degree) \textbf{\todo{(Figure X)}}. Each of this positions is tested regarding three different height level of the Kinect: 80 cm, 160 cm, and 240 cm. Therefore it is attached on a tripod like seen in \textbf{\todo{Figure X}}. At the end nine different combinations are covered to track a user on a slackline, which gives a good correlation of the Kinect position to the slackline. The testing person was recorded via \textit{KinectStudio}~\footnote{\url{https://developer.microsoft.com/de-de/windows/kinect/tools}}. With this one is able to record clips of a person tracked by the Kinect. Within the clips the joints of the skeletal tracking can be observed and reviewed. In the following the results discuss the feasibility and appropriate tracking positions.

\subsubsection{Slackline positioning}
\textit{\textbf{Sideways}}

The slackline positioned sideways in 90 Degrees rotated to the Kinect v2. The advantage of this is that the whole body on the slackline is in a constant line within the tracking area of the Kinect v2. With this no interference regarding the tracking distance can happen \textbf{\todo{(Figure X)}}. But the user tracking is very bad regardless of the Kinect height. This is because many body parts overlay and the Kinect v2 has problems to detect the body joints with appropriate accuracy and precision, which can be seen in \textbf{\todo{Figure X}}.

\textit{\textbf{Diagonal}}

The slackline stays diagonal in 45 Degrees to the Kinect. The frontal and end point of the slackline are now different in the vertical axis but it fits well in the tracking range \textbf{\todo{(Table X and Figure X)}}. 
%This could even result in a better trackability in matter of the depth field range, since the distance in the front shrinked and is therefore closer to the Kinect depth view. 
It shows better results in trackingability regarding the sideways positioning. Many body party does not occlude entirely but this problem is not entirely solved. It occurs that the joints of the arms and the body interfere with each other, especially at the end of the line. Also both legs occlude while stepping forwards \textbf{\todo{(Figure X)}}. This results in a bad skeletal tracking and depending on the executed exercise it can lead to detection problems.

\textit{\textbf{Frontal}}

Here the slackline is positioned frontal towards the Kinect. This resulted in the best user tracking out of the three positions. The sensor can see the full body and have nearly no problems with occlusions. Problems can occur at the starting position of the slackline since the slackline uses the entire depth range of the Kinect and the user is therefore close to the outermost range of it \textbf{\todo{(Figure X)}}.  A minor but non critical detection problem could occur with overlaying feets if the slacker stays with both on the line \textbf{\todo{(Figure X)}}.

\subsubsection{Comparing different Kinect heights}
Three main height levels were used to show the main differences of the tracking behaviour from the Kinect. It is mounted on a tripod and covers the heights of 0.80, 1.60 and 2.40 meters off the ground. 

Beginning with a height of 2.40 meters the Kinect has a very steep angle to track the slackers body on the full range of the slackline. Because of this the depth range shifts into the front like seen in \textbf{\todo{Figure X}}. Therefore if the slacker begins at the starting position on the slackline, she immediately reaches the end of the tracking area which can cause tracking problems. Also the closer she walks towards the Kinect the more occlusion happens regarding the joints.

With a height of 1.60 and 0.80 meters the entire body is fully visible in almost all ranges. The Kinect has a relatively flat angle. Because of this the slackline has to be positioned a little bit further away as former such that the body is fully visible on the entire line. This results in a more homogeneous depth range ( \textbf{\todo{Figure X}}). Problems can occur at the very end of the slackline depending on the slacker’s height. Her head or more body parts will be cropped. Therefore the slackline has to be slightly further away from the Kinect camera than on other heights. But since beginner will use the entire slackline range it can be neglected.

Overall a range between 0.80 meters up to 1.60 meters seems like the best height for the Kinect to track a person on a slacker. The tracking and view is more homogeneous and the angle flatter, which results in the possibility to use the full depth range. With a higher attachment the angle will be too steep, which results in less depth range, as well as more occlusions of body parts can occur.

\subsection{Best positioning for beginner learning purposes}
The best combination is placing the slackline frontal and having a Kinect height of 0.80 up to 1.60 meters. The Kinect can track the entire body and nearly no occlusion occur. Since only beginners are the main focus of this study, the starting position of the slackline plays an important role. Therefore for tracking purposes it is better to move the slackline closer to the camera. With this the first quarter of the slackline is cropped out of the view but the slacker can be tracked with a higher confidence at the starting position \textbf{\todo{(Figure X)}}.

\todo{\textbf{table}}
\section{Execise Integration}\label{5_2_gestureConstruction}
%- Recording of gestures --> Kinectstudio --> Making/Train gestures --> Visual gestures builder
The system should guide the user through predefined exercises for learning slacklining. To give feedback in an appropriate manner the exercises are recorded as custom gestures. The \textit{Kinect for Windows Human Interface Guidelines} describe the term \textit{gesture} as follows: "[...] we use the term gesture broadly to mean any form of movement that can be used as an input or interaction to control or influence an application."~\cite{MicrosoftHIG2014-mh}.
There are two approaches of creating custom gestures. The first is one is to do heuristics, which means to manually track the position of each joint and write conditions according to the action that should happen if the joints exceed a threshold or are is in a defined range. This is used and implemented for simple gestures like raising the hand over the head. For more complex gestures, the developer must have a good understanding about how the human body behaves and moves. In the most cases developer have not the appropriate expertise. Hence it is recommended to use the Visual gesture builder (VGB) provided by Microsoft for more complex gestures.
%any form of movement that can be used as an input or interaction to control or influence an application. Gestures can take many forms, from simply using your hand to target something on the screen, to specific, learned patterns of movement, to long stretches of continuous movement using the whole body.

\subsection{Visual gesture builder}
This tool relies on machine learning and looks at the data given by the developer via pre recorded clips. With these it builds a database that can then be used to track the actual gesture in an application. The more data is provided to it the better the detection by the Kinect. Another advantage is that environmental factors are not that complex to handle as in comparison to heuristics. For example if the sensor is set too high or too low the developer has to consider this in his code and it can blow up managing and maintaining such factors in code. With the VGB the developer just records data with the sensor on the appropriate height and let the machine learning algorithm learn it. The cons are the huge file size of the recorded clips which can take very much disk space. Also setting the keyframes for parts that the builder should detect is time consuming whereas on the other hand it is simple and user friendly.

\subsection{Building gestures workflow}
The workflow for creating a gesture is almost always the same (Figure~\ref{fig:5_3_gestureCreation}). 
First the actual gesture has to be recorded via KinectStudio. This is a tool provided by Microsoft for monitoring and recording clips of the Kinect streams. After finishing with the recording a new project can be made in the Visual Gesture Builder. The developer selects the body parts that are necessary for the gesture. After that the indicator of the gesture has to be defined, i.e. if it is discrete or continuous. Discrete means that the system determines if a gesture is currently performed or not. It provides a confidence value that determines the correctness of the persons execution regarding the specific gesture. This is the majority for gesture tracking like e.g. raising the hand or lifting a leg. However the continuous indicator means that the progress of a gesture can be measured. Often multiple small gesture are combined to an entire gesture. This could be for example a golf swing or switching the standing leg, where rather the progress has to be measured than the confidence~\cite{MicrosoftVGB}.

\begin{figure}[htb]
	\centering
	\begin{minipage}[t]{1\linewidth}
		\centering
		\includegraphics[width=1\linewidth]{Pictures/5_3_gestureCreation}
		\caption{Workflow of creating a gesture database}
		\label{fig:5_3_gestureCreation}
	\end{minipage}
\end{figure}

After the project creation training data can be inserted, which are the prior recorded clips. The developer has to define tags that describe the starting and the end point of the gesture. After finishing with the tagging a gesture database file can be built. Afterwards it can be analysed via a live preview or with other recorded clips in a separate analysing area. If some misbehaviour appears the developer has to record and add more clips for the gesture or the tags have to be adjusted. Lastly after the testing phase a gesture database file can be built and then implemented in the application for gesture detecting. %The structuring of the application architecture is part of the next section.
\section{System architecture}\label{5_2_systemArchitecture}
%- System architecture of system --> Unity3D, Kinect SDK, Kinectstudio, VGB --> kinect sdk free to use since version X
In the following several components of the general system architecture will be described that that are necessary for the functionality of the interactive learning system with real-time feedback and for the study afterwards. An overview can be seen in figure \todo{systemarchitecture}.

\subsubsection{Hardware and software components}
% Kinect, Beamer, Screen, Slackline --> Alpidex High Performance

As slackline the mobile \textit{alpidex POWER-WAVE 2.0} is used. It is attached in front of the \textit{Microsoft Kinect v2}, which is used as tracking device. The Kinect itself is attached on a \todo{modell} tripod with a height of about \todo{height, hüfthöhe?}. A \textit{Steambox PC} \todo{footnote specs} was sufficient to fulfil the recommended specs of the Kinect: \todo{kinect specs}. To give the user a more immersive feeling a projector \todo{modell} with a resolution of 1920x1080 that is attached on a traverse system with a projection screen of size \todo{2x3m} is used.

% KinectStudio, VGB, Unity3D, Kinect SDK for unity, Kinect MS-SDK
For software realization the cross-platform game engine \textit{Unity3D} by \textit{Unity Technologies} is used. Applications developed with this can be deployed for several platforms like e.g. the most known Windows, Linux, macOS, Android, iOS, etc. For accessing the data stream of the Kinect the \textit{Microsoft Kinect SDK v2.0}
\footnote{\label{fn:kinectTools}\url{https://developer.microsoft.com/de-de/windows/kinect/tools}}has to be installed on the PC. Microsoft offers also a \textit{Kinect for Windows Unity package}\cref{fn:kinectTools}, which delivers all required scripts to manage the data stream in Unity for creating a Kinect based unity application. Since \textit{Unity 5} it can be used with the free personal edition, whereas before it could be only used with the pro version. Also the \textit{Kinect v2 Examples with MS-SDK} \footnote{\url{https://www.assetstore.unity3d.com/en/\#!/content/18708}} by Rumen Filkov were used for making data access and interaction implementation easer as well as getting an idea on how to handle incoming data from the Kinect.

\subsection{Software implementation}
The software development process consists of the interplay of two system components. First Unity itself, which is used to display and manage actions by the user on the interface. Second the Kinect SDK plugin for accessing the users action recognized by the Kinect device. In the following the interaction integration of the system are further described.

%This application a 3D project has been generated, since the Kinect uses 3D space to track the user, she should be able to interact within that, and a 3-Dimensional environment design is considered. Further some examples of the interaction implementation is described.
\subsubsection{Example \& Engagement gesture}
An example on how to manage a raising hand script can be seen in listing~\ref{lst:codeEngagement}. The \textit{KinectManager} exists as a Game Object in the scene and has to be referenced in the script. One has to assure that the Kinect is initialized, a user has been detected, and the relating joints are detected. In line~\ref{lst:codeEngagement15} the condition checks if the current vertical position of the right hand is greater than the current position of the head joint. If this is the case, the next scene is loaded. This is actually part of the script for the first scene in the application, in which the user has to engage with the Kinect by doing the described movement.

\begin{lstlisting}[caption=C$^\sharp$ example code for tracking a raising hand, label=lst:codeEngagement]
if (_kinectManager && 
	  _kinectManager.IsInitialized() && 
	  _kinectManager.IsUserDetected())
{
	long uId = _kinectManager.GetPrimaryUserID();

	if ((_kinectManager.IsJointTracked(uId, (int) _jointHandRight) && 
		   _kinectManager.IsJointTracked(uId, (int) _jointHead))
	{
		Vector3 jointPosHandRight = 
		 _kinectManager.GetJointKinectPosition(uId, (int) _jointHandRight);
		Vector3 jointPosHead = 
		 _kinectManager.GetJointKinectPosition(uId, (int) _jointHead);			

		if (jointPosHandRight.y > jointPosHead.y)(*@ \label{lst:codeEngagement15} @*)
			SceneManager.LoadScene("Tutorial");
	}
}
\end{lstlisting}

\subsubsection{Hand cursor as interaction device}
Specifically for the cursor, with which the user can interact with elements on the system by pushing the hand towards the kinect, the state will be clarified by a circle like seen in figure \todo{insert figure below}.

The state of the current interaction is visualized properly by providing a circle around the hand cursor that represents the progress like seen in figure \ref{fig:handcursorProgress}.
\begin{figure}[htb]
	\centering
	\begin{minipage}[t]{1\linewidth}
		\centering
		\includegraphics[width=0.6\linewidth]{Pictures/handcursorProgress}
		\caption{Progress of handcursor (Left: Default, Middle: In progress, Right: Finished)}
		\label{fig:handcursorProgress}
	\end{minipage}
\end{figure}

\begin{comment}
- Kinect used for tracking --> how Kinect tracks user - skeleton, infrared, own algorithm -> RW

- Technical feasibility in here?

- Recording of gestures --> Kinectstudio --> Making/Train gestures --> Visual gestures builder

- System architecture of system --> Unity3D, Kinect SDK, Kinectstudio, VGB --> kinect sdk free to use since version X

- Data management --> json file, default exercise json and each user has its own json file 

- Engagement with Kinect

- Interaction components (track hand joints, PHIZ, Constraints) --> different interactions tested (closing hand, V-sign, hover, pushing) --> not all good because of distance to Kinect --> In unity addon written for managing data --> create new users, load exercise in user, adjust jsons

- tier and exercises as level design --> locked and unlocking by successfully accomplishing exercise

- Integration VGB databases in Unity and how to track gestures in it --> gesturedetector, eventlistener

- Providing feedback properly --> confidence/progress of gestures in event listener, checklist (joint detection), 
--> user viewer --> how it works (making cloudy map regarding user position, drawing lines for skeletons)

- Summary screen --> feedback about prior or entire tier performance with time, confidence and attempts

- User interface design --> sketches, mock ups, development --> workflow figure
\end{comment}

\section{User Interface}\label{5_4_userInterface}
The user should be introduced to the stage. In here the purpose, goal, and helpful techniques should be given, such that the user becomes an overview about the exercises. At last a summary scene shows several performance parameter for the exercises in this stage.

The user starts with an engagement gesture like raising her hand over the head to convey that the system initially recognises and responds to a user action. After that a tutorial about the interaction with the system will be given that covers clicking and scrolling techniques. Now she's confident with the system interaction and can select a profile in the user select to train. This loads the profile which leads to the stage selection menu. In here she can select a stage, whereas initially the first one is can be selected and the others have to be unlocked by successfully accomplishing all exercises in the preview stage. Selecting a stage leads to the exercise menu. In here she has to read initially the stage introduction to become a basic understanding about the exercises in here. After reading this, it unlocks the first exercise. Selecting an exercise leads to the side selection, where the user has to choose the side she wants to train for this exercise. This is followed by an introduction of the exercise, in which is explained how to perform it correctly. If the user is ready, she should stay in a starting position to be able to start the exercise execution. In here she find all relevant elements to perform the exercise, like indicators for the time, repetitions, confidence and a checklist, which helps her to correctly execute the exercise. After successfully executing the exercise, a summary is shown which summarizes the user performance. Then she can return to the main menu or directly approach the next exercise. A stage summary gives an overview about all exercises with average performance parameters.

\todo{replace figure with directional flow}
\begin{figure}[htb]
	\centering
	\begin{minipage}[t]{1\linewidth}
		\centering
		\includegraphics[width=0.8\linewidth]{Pictures/conceptScenarioFlow2}
		\caption{Scenario workflow}
		\label{fig:scenarioWorkflow}
	\end{minipage}
\end{figure}
