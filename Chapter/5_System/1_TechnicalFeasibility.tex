\section{Technical feasibility}\label{5_1_technicalFeasibility}
Tracking a person on a slackline is very different from a regular situation.
The interplay between the range of the slackline, the movement of the line itself, unpredictable movements of the user, and his balancing actions could disturb the Kinect tracking performance. This would lead to imprecise and inaccurate tracking data. Furthermore there exists no comparable work about how to track user properly on a slackine with the Kinect.

The major point is to compare different slackline positions and rotations regarding multiple angles and heights of the Kinect. This will clarify how good a person can be tracked on a slackline and which is the best combination of the slackline and Kinect positioning.

\subsection{Constraints of the Kinect} 
A considerable role plays the angle and detection range of the Kinects depth sensor regarding the length of the slackline. The sensors’ angle of vision covers in horizontal 70 degrees and in vertical 60 degrees (Figure~\ref{fig:5_1_1_visionAngle}). Hence, tracking the users' height could result into problems because the slackline is about 30 cm off the ground, which in combination with a person on a slackline results in a higher body positioning of the person. The total tracking range of the sensor lies between 0.5 and 4.5 meters, whereas the sweet spot area lies between 1 up to 4 meters~\cite{MicrosoftHIG2014-mh} (Figure~\ref{fig:5_1_1_trackingRange}). Since the mobile slackline device has a length of three meters, it would fit entirely into the sweet spot tracking range. To track user for further training on a longer slackline, the depth range is not sufficient. This could be solved by using more than one Kinect device to enlarge the range. 
\begin{figure}[htb]
	\centering
	\begin{minipage}[t]{0.44\linewidth}
		\centering
		\includegraphics[width=1\linewidth]{Pictures/5_1_1_visionAngle}
		\subcaption{Angle of vision}
		\label{fig:5_1_1_visionAngle}
	\end{minipage}
	\hfill
	\begin{minipage}[t]{0.44\linewidth}
		\centering
		\includegraphics[width=1\linewidth]{Pictures/5_1_1_trackingRange}
		\subcaption{Kinect tracking range}
		\label{fig:5_1_1_trackingRange}
	\end{minipage}
	\caption{Sensor constraints of the Kinect v2~\cite{MicrosoftHIG2014-mh}}
	\label{fig:5_1_1_sensorConstraints}
\end{figure}

\subsection{Testing scenario}

%The test took place in the laboratory of the research group in the \textit{german reasearch center for artificial intelligence}~\footnote{\url{https://www.dfki.de/web/kontakt/dfki-saarbruecken}}. A big advantage of this is the large space to place the slackline easily in different variations. 
The slackline is placed in three positions to the Kinect: frontal (0 Degree), diagonal (45 Degree) and sideways (90 Degree) \textbf{\todo{(Figure X - 1)}}. Each of this positions is tested regarding three different height level of the Kinect: 80 cm, 160 cm, and 240 cm. Therefore it is attached on a tripod ( \textbf{\todo{Figure X - 2}}). At the end nine different combinations are covered to track a user on a slackline, which gives a good correlation of the Kinect position to the slackline. The testing person was recorded via \textit{KinectStudio}~\footnote{\url{https://developer.microsoft.com/de-de/windows/kinect/tools}}. With this it is able to record clips of a person tracked by the Kinect.
Within the clips the joints of the skeletal tracking can be observed and reviewed. In the following the results discuss the feasibility and appropriate tracking positions.

\subsubsection{Slackline positioning}
With a slackline positioned sideways in 90 Degrees rotated to the Kinect, the advantage is that the whole body on the slackline is in a constant line within the tracking area. No interference regarding the limits of the tracking range can happen \textbf{\todo{(Figure X)}}. But the user tracking is very bad regardless of the Kinect height. This is because many body parts overlay and the Kinect has problems to detect the body joints with appropriate accuracy and precision, which can be seen in \textbf{\todo{Figure X}}.

By placing the slackline diagonal in 45 Degrees to the Kinect, the frontal and end point of the slackline now differ in the vertical axis but it fit well in the tracking range \textbf{\todo{(Table X and Figure X)}}. 
%This could even result in a better trackability in matter of the depth field range, since the distance in the front shrinked and is therefore closer to the Kinect depth view. 
It shows better results in tracking ability than the former positioning, because many body parts does not occlude. But this problem is not entirely solved. It occurs that joints of the arms interfere with the body joints, especially at the end of the line. Also both legs occlude while stepping forwards \textbf{\todo{(Figure X)}}. This results in a relatively bad skeletal tracking and depending on the executed exercise it can lead to detection problems.

Positioning the slackline frontal towards the Kinect resulted in the best user tracking.
The sensor can see the full body and track the joints without any problems.
But since the slackline uses the entire depth range of the Kinect detection failure occurred at the starting position of the slackline. This is because the user stands closer to the outermost limit of the detection range~\textbf{\todo{(Figure X)}}.

\subsubsection{Comparing different Kinect heights}
%Three main height levels were used to show the main differences of the tracking behaviour from the Kinect. It is mounted on a tripod and covers the heights of 0.80, 1.60 and 2.40 meters off the ground. 

Beginning with a height of 2.40 meters the Kinect has a very steep view angle to track the slackers body on the full range of the slackline. Because of this the depth range shifts into the front like seen in \textbf{\todo{Figure X}}. Therefore if the slacker begins at the starting position on the slackline, she immediately reaches the end of the tracking area which can cause tracking problems. Also the closer she walks towards the Kinect the more occlusion happens regarding the joints.

With a height of 1.60 and 0.80 meters the entire body is fully visible in almost all ranges. The Kinect has a relatively flat angle. This results in a more homogeneous depth range ( \textbf{\todo{Figure X}}). Problems can occur at the very end of the slackline depending on the persons' height, which leads to cropped body parts like the head or arms. Therefore the slackline must be positioned slightly further away from the Kinect camera than on other heights.

\todo{\textbf{table}}
%The tracking and view is more homogeneous and the angle flatter, which results in the possibility to use the full depth range. With a higher attachment the angle will be too steep, which results in less depth range, as well as more occlusions of body parts can occur.

\subsection{Best positioning for beginner learning purposes}
% But since beginner will use the entire slackline range it can be neglected.
The best combination resulted placing the slackline frontal and having a Kinect height of 0.80 up to 1.60 meters. The Kinect can track the entire body with nearly no joint occlusion. Since the focus lies mainly on beginners of the study in this thesis, the starting position of the slackline plays an important role. Hence, for tracking purposes it is better to move the slackline closer to the camera. With this a little bit of slackline is cropped out of the view but the slacker can be tracked with a higher confidence at the starting position, which is more important \textbf{\todo{(Figure X)}}.
