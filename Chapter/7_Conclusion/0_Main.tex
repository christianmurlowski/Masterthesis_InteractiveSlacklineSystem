\chapter{Conclusion and Future Work}\label{7_conclusion}
This last chapter gives a summary of the concept, system integration, and conducted study.
Further it provides possible ideas to extend and improve the system itself as well as recommendations for further research work.

\section{Conclusion}
At first will be discussed how the research goals stated in section \textit{\nameref{1_2_researchGoals}} have been realized within the scope of this thesis.

\subsubsection{Exploration of slackline training and designing an appealing interactive system}
Slacklining shows positive effects on muscle strength improvements, muscle activation, as an alternative way to general balance training, and general improvements in postural control.
The Microsoft Kinect resulted in the best tracking device for the case of this thesis.
Appropriate and simple feedback help the user during the exercise execution as well as motivating him for further exercises with an enjoyable but challenging virtual training environment and a user friendly interface.


\subsubsection{Conceptual design of a prototypical interactive slackline learning system}
The conceptual design has been elaborated by respecting Nielsens ten usability heuristics and defining basic and system specific requirements based on the related work and comparable applications.
Splitting exercises into levels allows to separate them into different difficulty ranks and unlocking them by the user through accomplishing exercises gives a sense of motivation and gameplay.

\subsubsection{Implementation for the usage in a user study}
The system has been built with the help of Unity3D and Microsoft Kinect SDK as connection between Unity and Kinect hardware.
For displaying the UI on a  big screen a projector has been used.
The best combination of positioning resulted in a Kinect height of 0.80 up to 1.20 meters and positioning the Slackline directly to the Kinect.
For the movement recognition the microsoft tools \textit{KinectStudio} and \textit{Visual Gesture Builder} have been used.

\subsubsection{Investigation of the slackline learning system}
A conducted study compared the slackline learning system with a personal human trainer.
A 2 (group: ISG, HTG) x 2 (time: PRE, POST) mixed factorial study design was chosen with the measurement parameters single leg stance, steps walked over the line, and distance walked over the line.
The results showed no interaction effect between the groups for the time and no main effect between subjects.
However a significant main effect for the time has been found.
This means that both groups do not differ after the training regarding their performance, but all participants improved their own balance skill in almost all categories.

\section{Future Work}
The following will include ideas and approaches for further development with the system from multiple perspectives and application fields.

Further work could investigate the usage of the slackline learning system as a home workout system.
This could show if it would be an appropriate alternative for private persons to become more fit and motivate them for balance sport activities.

Another interesting part would be the investigation of the slackline learning system in fields of sport medicine like e.g. for patients in physiotherapy.
Patients could be more motivated through the gamified environment and for unlocking exercises, which helps to support them for curing their disorder.
Because of its autonomous usage it would also ease the workload the medical personal at the same time, so they could focus on other work than supervising patients during their training period.

Noted by participants in the study of this thesis, the system could find its application as balance training system for several sport activities.
Especially where self control and controlling the own body balance is essential like in martial arts, dancing, or rowing.
Furthermore gyms rely more and more on a virtual trainer.
Therefore such a system could find its application also in a gym or in a climbing hall as alternative balance training method.

The system itself could also be improved to make it ore attractive and enhance the user experience with it.
Especially for people that have a sense of competition or want to improve themselves for this kind of sport. 
To achieve this the system could first assess the skill level of the user with pre defined exercises.
Based on the outcome a pool of exercises would be created by the system.
Like seen in section~\textit{\nameref{3_3_1_learningConcepts}} a differential method could also serve as an alternative training method to give the user more freedom in choosing exercises of a certain skill area.
The skill level of the user would then be calculated in points based on her performance, which could be the amount of attempts for the exercise, how confident she was, and how much time she needed to accomplish the exercise.
Furthermore an achievement system could be used to motivate the user and reward her performance.
The videos could be replaced by an animated character, which supports the trainee during the exercises like seen in the \textit{Nike +} application of section~\textit{\nameref{1_1_motivation}}.

Further it would be interesting to adapt other sport activities than slacklining for such a learning system.
Participants of the study named for example boxing, basketball, yoga, dancing, or skiing.
In general all balance activities or activities in which a choreography are part of them.

Another open question would be, if slacklining could be combined with virtual reality technology.
This would be an application for more advanced slacker, because the risk of accidents would be too big for beginners on a slackline.
In special, elaborating the the impact of a pleasing virtual environment on the slacker and if it could provide a more immersive feeling because of the virtual environment.
Further the learning approach could also be integrated in virtual reality, in which a virtual trainer provides the exercises and demonstrates them to the trainee in a virtual environment

% gamified home workout system
% 


% study
% longer with both groups
% balance skill test and defining groups concerning balance skill
% more advanced exercises for longer study --> going backwards and avoiding obstacles
% dynamically exercises like walking on the line