\chapter{Conclusion and Future Work}\label{7_conclusion}
This last chapter provides a conclusion of the research questions.
In the second part, it states issues with the system that should be improved.
Furthermore, possible ideas to extend and improve the system itself as well as recommendations for further research work will be provided.

\section{Conclusion}
At first the research questions stated in section \textit{\nameref{1_2_researchGoals}} will be discussed and how they have been realized within the scope of this thesis.

\subsubsection{\nameref{rq_slacklineTrainingSystem}}
Slacklining shows positive effects on muscle strength improvements and muscle activation.
It also can be used as an alternative way to general balance training, because it shows improvements in postural control.
The Microsoft Kinect resulted in the best tracking device for the case of this thesis.
Appropriate and simple feedback help the user during the exercise execution, as well as motivating her for further exercises.
This is achieved by an enjoyable virtual training environment, a challenging exercise structure, and a user friendly interface.


\subsubsection{\nameref{rq_concept}}
The conceptual design has been elaborated by defining basic and system specific requirements based on the related work and comparable applications, Nielsens ten usability heuristics, and human interface guidelines provided by Microsoft.
Splitting exercises into levels allows to separate them into different difficulty ranks.
Unlocking levels and exercise through accomplishing them, gives a sense of motivation and enjoyable gameplay.
Providing the user with a well structured interface and a clear workflow helps to enhance the user experience

\subsubsection{\nameref{rq_development}}
The system has been built with the help of Unity3D and Microsoft Kinect SDK as connection between Unity and Kinect hardware.
For displaying the UI on a  big screen a projector has been used.
The best combination of positioning the Kinect and the slackline resulted in a Kinect height of 0.80 up to 1.20 meters and positioning the slackline directly in front of the Kinect.
Exercises for the movement recognition were created with the tools \textit{KinectStudio} and \textit{Visual Gesture Builder} by Microsoft.
Furthermore, checking on small and simple gestures has been achieved with heuristics, which means programmatically checking on certain joint positions and angles in the x-, y- and z-axis.

\subsubsection{\nameref{rq_study}}
A conducted study compared the slackline learning system with a personal human trainer.
A 2 (group: ISG, HTG) x 2 (time: PRE, POST) mixed factorial study design has been used with the measurement parameters single leg stance, steps walked over the line, and distance walked over the line.
The results showed no interaction effect between the groups for the time and no main effect between subjects.
However, a significant main effect between pre and post measurement has been found for all participants.
This means that both groups do not differ after the training regarding their performance, but all participants improved their own balance skill in almost all measurement categories.
Considering the numeric data results of the learning progress, the performance of both groups (HTG and ISG) can be seen as very similar.
Therefore the learning progress on the slackline learning system can be compared with a personal human trainer.

\section{Limitations and Future Work}
The following section will suggest improvements, ideas, and approaches for further development with the system from multiple perspectives and application fields.

For further development of the system in the area of slacklining the tracking should be improved.
Like seen in section \textit{\nameref{results_interview}} there exist problems with tracking exercises while going up and walking steps on the line.
More variations of the exercises should be tracked with several people to enhance the gesture database.

The SLS itself can be improved to make it more attractive for different user groups.
Especially for people that have a sense of competition or want to improve themselves for this kind of sport. 
To achieve this, the system could at first assess the skill level of the user with predefined exercises.
Based on the outcome, a pool of exercises would be created by the system.
Like seen in section~\textit{\nameref{3_3_1_learningConcepts}}, a differential method could also serve as an alternative training method to the methodical routine used in this study.
An integration of a differential method provides the user with more freedom in choosing exercises over the entire range of the difficulty levels.
The experience of the user could also be calculated in points based on her performance, which could be the amount of attempts for the exercise, how confident she was, and how much time she needed to accomplish the exercise.
Furthermore, an achievement system could be tested to motivate the user and reward her.
An animated character would be an alternative to the looping videos in the exercise instruction.
It could be used to support the trainee during the exercises, like seen in the \textit{Nike +} application of section~\textit{\nameref{1_1_motivation}}.

In terms of application scenarios, further work could investigate the usage of the slackline learning system as a home workout system.
This can show if the system would be an appropriate alternative for private people to become more fit and motivate them for balance sport activities.
A similar scenario would be to integrate the SLS in a company in which office job is the main activity.
It would be interesting, if employees are motivated to use the system for having a certain amount of variety in their job.
Additional research could observe if employees will become more fit, prevent problems with their posture, or improve their posture, which is strained due to sitting activities.

Another interesting part would be the investigation of the SLS in fields of sport medicine, e.g. for physiotherapy or in a rehabilitation center.
Patients could be more motivated through the gamified environment and for unlocking exercises.
A further question would be, if it has positive effects to support the patients in respect of the cure of their balance disorder.
Because of its autonomous usage a workload reduction of the medical personal can be assumed.
They could focus on other work than supervising patients during their training period.

Noted by participants in the study of this thesis, the system could find its application as balance training system for several other sport activities.
Especially where self control and controlling the own body balance is essential, like in martial arts, dancing, or rowing.
Furthermore, gyms rely more and more on a virtual trainers.
Such a learning system could therefore find its application in a gym environment or in a climbing hall as alternative training method.

It would also be interesting to adapt other sport activities than slacklining for such a learning system.
Participants of the study named for example boxing, basketball, yoga, dancing, or skiing.
In general, all balance activities or activities in which a choreography are part of them could be integrated.

Another open question would be, if slacklining could be combined with virtual reality technology.
This would be an application for more advanced slacker, because the risk of accidents is probably be too high for beginners on a slackline.
In special, elaborating the impact of a pleasing virtual environment on the slacker and if the application could provide a more immersive feeling because of the virtual environment.
Further, the learning approach could also be integrated in virtual reality, in which a virtual trainer provides the exercises and demonstrates them to the trainee.

% gamified home workout system
% 


% study
% longer with both groups
% balance skill test and defining groups concerning balance skill
% more advanced exercises for longer study --> going backwards and avoiding obstacles
% dynamically exercises like walking on the line