\section{General setup of the study} %Mobile slackline device and constraints with the Kinect v2 vamera view
One essential part that is needed for the experiment is the slackline. But it exists in many different forms and variations as seen in \textbf{Chapter three} \textbf{ (Figure X)}. Also all lines have to use a fixing mechanism and are normally attached on a tree, pole, pillar, or with anchors on the ground or on a wall. In the case of this study, it would result in a constraint of variability. 
The choice felt therefore on a mobile slackline device variation, which is the most suited alternative. It consists of a slackline itself that is tensed around brackets at both ends. For feasibility reasons and because the focus of this scenario lies mainly on beginners, the device is comparatively short. A variable middle rail can be telescoped and vary the length of the device from \textbf{1 m up to 3.5 m}. With this it is possible to test it indoors and in different positions with a minimum of effort \textbf{(Figure X)}. Another advantage of this is the independence and variability of the device. This makes it easy to test it for the best position regarding the tracking camera, which is another essential part of the experiment.

In slacklining the user should be free in his movement and match predefined gestures. Therefore the low-cost tracking camera Kinect v2 is used as tracking device. As mentioned in \ref{interactiveTechnology} this is the most appropriate one out of the available user tracking devices.
A mentionable role plays the detection range of Kinect’s depth sensor regarding the length of the slackline. The sensors range lies between 0.5 up to 4.5 meters \textbf{[CITE]}. Since a mobile slackline is used with a length up to 3.5 meters, it would fit entirely in the tracking range. To track user for further training on a longer slackline, the depth range is not sufficient. This could be solved by using more than one Kinect device to \textbf{enlarge} the range. 
\\
Generally a major point for tracking the user is the interplay between positioning the tracking device and slackline The coherence of angle and height of the Kinect v2 is essential for the depth range, which varies by changing these parameters. This will be discussed in the following.