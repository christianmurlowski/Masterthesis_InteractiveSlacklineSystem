\section{Introduction}

Before starting with a concept of the assistance system a few questions have to be clarified. First if it is possible to track a human body on the slackline with the Kinect v2. Is the answer positive than the second question should be how good is the tracking behaviour of the device and can it therefore be used to track a human body on a slackline.

In many movement scenarios in balance training, which are similar to slacklining, the user were successfully trackable like seen in several related work \textbf{[CITE]}. Hence the expectation is that the Microsoft Kinect v2 should be able to track the human body on a slackline with an appropriate accuracy and precision. But the range of the slackline, the movement of the line itself, unpredictable movements of the user, and his balancing actions could also possibly disturb the tracking ability. This can then lead to imprecise and inaccurate tracking data that negate the stated findings of other tracking balancing scenarios.

With this in mind multiple angles, positions of the camera, as well as the slackline have been tested. This Chapter will therefore describe a feasibility study, which gives clarification about the named questions. 

