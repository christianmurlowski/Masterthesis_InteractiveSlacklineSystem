\section{Introduction into slacklining}\label{3_1_introductionSlacklining}
The term slackline has its origin in the 1980's. Some climbers balanced on a tubular webbing in contrast to the existing balance activity tightrope, where you balance on a steel rope. Therefore they used the term \textit{slack wire} that later transformed into \textit{Slackline}, which means loose line~\cite{Zak2011-sl, Balcom2005-wl, MillerMauser2013-sl}.

Hence slacklining comes from the climbing sport and in a broader sense it can be compared with ropedancing~\cite{Kleindl2011-bl}. The line itself is made out of a nylon ribbon. Unlike in ropedancing the ribbons width is between 2.5 and 5 cm and is very flat. To fixate the line two stable fixation points are needed. Mostly it is then tensed between these points with a tension device, which is normally a part of the slackline. ~\cite{Kleindl2011-bl}. Because of the nylon texture the line will expand under pressure, if someone stands on it. Therefore it is very dynamic and the person has to outbalance every sway~\cite{Kroiss2007-ab}. Given this elasticity a person can e.g. bounce, bob, or swing on the line with which many application scenarios has resulted~\cite{Balcom2005-wl}.

\todo{phylosophy evtl erläutern}