\section{Introduction into Slacklining}\label{3_1_introductionSlacklining}
The term \textit{slackline} has its origin in the 1980's. In contrast to the existing balance activity tightrope, where you balance on a steel rope, some climbers balanced on a tubular webbing. Therefore they used the term \textit{slack wire} that later transformed into \textit{slackline}, which means loose line~\cite{Zak2011-sl, Balcom2005-wl, MillerMauser2013-sl}.

Hence slacklining comes from the climbing sport and can be compared with ropedancing in a broader sense~\cite{Kleindl2011-bl}. The line itself is made out of a nylon ribbon. Unlike in ropedancing the ribbons width is between 2.5 and 5 cm and very flat. It has to be tensed between two stable fixation points like trees, stable pillars, fixation systems on the ground, so called \textit{A-Frames}, or on a rock with a bolt hanger and carabiner. Mostly this is done with a tension device, which is in general a ratchet or pulleys depending on the fix points~\cite{Kleindl2011-bl}. Because of the nylon texture the line will expand under pressure once someone stands on it. Given this elasticity makes it very dynamic and the slacker has to outbalance every sway~\cite{Kroiss2007-ab}. To be in control of her body behaviour she has to act very calm, which makes slacklining in general a quiet and meditative sport activity. Besides walking one can also e.g. bounce, bob, or swing on the line. As a result various fields of application arose from this variability, which is further described seen in the next section.