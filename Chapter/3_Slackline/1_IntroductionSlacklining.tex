\section{Introduction into slacklining}\label{3_1_introductionSlacklining}
\subsection{What is slacklining?}
The term slackline has its origin in the 1980's. Some climbers balanced on a tubular webbing in contrast to the existing balance activity tightrope, where you balance on a steel rope. Tehrefore they used used the term \textit{slack wire} that later transformed into \textit{Slackline}, which menas loose line~\cite{Zak2011-sl, Balcom2005-wl, MillerMauser2013-sl}.
\textbf{ \todo{picture of slacklining maybe myself}}

Slacklining itself comes therefore from the climbing sport and in a broader sense it can be compared with ropedancing~\cite{Kleindl2011-bl}. The line itself is made out of a nylon ribbon. Unlike in ropedancing the ribbons width is between 2.5 and 5 cm and is very flat. To fixate the line two stable fixation points are needed. Mostly it is then tensed between these points with a tension device, which is normally a part of the slackline. ~\cite{Kleindl2011-bl}. Because of the nylon texture the line will expand under pressure, if someone stands on it. Therefore it is very dynamic and the person has to outbalance every sway~\cite{Kroiss2007-ab}. Given this elasticity a person can e.g. bounce, bob, or swing on the line with which many application scenarios has resulted~\cite{Balcom2005-wl}.

Further depending on the length, tension or height a few slackline variations have originated~\cite{MillerMauser2013-sl, Kleindl2011-bl, Thomann2017-ab}. Regarding the height one can differentiate between a \textit{lowline} and a \textbf{highline}. The former is the category in which almost all lines match because it describes a height in which one can safely jump off the line. On a \textit{highline} this is not possible. Here you have to make safety provisions like a seperate system where the person can hook herself in this system above or under the regular line~\cite{Kleindl2011-bl}.
\todo{picture lowline highline}

The following terms describe some categorization of the slackline in different application scenarios, which can differ in its scenario or blend into each other. The simple \textit{trickline} is the common slackline. It is tensed a bit loose in about the height of the knees and has a length up to 30 m. A \textit{jumpline} is stronger tensed to make jumps on the line possible. They have a length of 8 - 14 m and are a bit higher than the trickline. With a \textit{rodeoline} the line is actually more slacked and and has the highest amplitude. It is a relatively short line with a length of 5 - 8 m and the fixation points are in about 2 m such that if a person stands in the mid of the line it is just about above the ground and can swing on it. Slacklines beyond 30 m are called \textit{longline}. The goal here is to walk as far as possible without falling off the line. Beside these there exist also \textit{waterlines}, which is simply a slackline tensed over water and an \textit{uranline} where manmade or urban structures are used to tense the line between.
\todo{picture of each line}

\begin{comment}
\begin{itemize}
\item What is it actually
\item „The fascination behind it“ —> Optional?
\item Different ways to slackline \textbf{ \todo{figures}}

\begin{itemize}
\item Trickline/Normal --> up to 30m, loose tensed
\item Jumpline 	--> up to 30 m, hard tensed
\item Longline --> more than 30m but no constraint in length
\item Highline --> up in the height and long
\item Rodeoline --> loose tensed
\item Waterline --> over water

\end{itemize}

\item Approach scenarios
\begin{itemize}
\item Hobby
\item Competition
\item Tricklining
\item Walking
\end{itemize}

\item Tension
\begin{itemize}
\item Strong
\item Loose
\end{itemize}

\end{itemize}
\end{comment}