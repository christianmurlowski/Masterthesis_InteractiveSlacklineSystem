\section{Summary}\label{4_7_conclusion}
The interactive slackline learning system should be able to teach and support users how to slackline with predefined exercises. For proper realization the SLS has to consider several aspects. It should comprise an appropriate amount of user experience. By following and respecting Nielsens ten usability heuristics it provides an overall standard of usability. Further, some system specific groundwork should be integrated. This involves, for example, autonomous interaction, adequate user tracking, and supportive real-time feedback. More specifically, the levels and exercises should be able to unlock by successfully completing exercises. Each of these exercises should be introduced to the user to give her an understanding of the correct execution. Lastly, the feedback system is one of the biggest components. It involves audiovisual real-time feedback for the user interaction as well as exercise execution, and general feedback for rating her performance regarding an exercise or the entire level. The next chapter \textit{\nameref{5_systemIntegration}} relies on this concept and discusses the development process.
%The next chapter elaborates the \textbf{\nameref{5_systemIntegration}} and development of this system regarding the concept .