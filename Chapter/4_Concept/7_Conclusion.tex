\section{Conclusion}\label{4_7_conclusion}
The interactive slackline system should be able to teach and support user how to slackline with predefined exercises. For proper realization a few things have to be considered. The system should comprise an appropriate amount of user experience. By following and respecting Nielsens ten usability heuristics it provides an overall standard of usability. Further some system specific groundwork should be integrated. This involves for example autonomous interaction, proper user tracking, and supportive real-time feedback. More specifically the stages and exercises should be designed as a level like integration that can be unlocked by successfully completing exercises. Each of the exercises should be introduced to the user to give her an understanding of the correct execution. Lastly the feedback system is one of the biggest component. It involves audiovisual real-time feedback for the user interaction as well as exercise execution, and general feedback for rating her performance regarding an exercise or the entire stage. The next chapter \textit{\nameref{5_systemIntegration}} relies on this concept and elaborates the development process.
%The next chapter elaborates the \textbf{\nameref{5_systemIntegration}} and development of this system regarding the concept .