\section{Feedback System}\label{4_5_feedbackSystem}
%The user should get feedback about her performance during exercise execution. 
%Another big role of this plays in the exercise execution. Here the user interacts with the system to match a predefined gesture for accomplishing the exercise. Therefore real time feedback is provided which gives her hints about the right interaction and how good it is performed. More information about the feedback methods can be found in feedback system

% - https://www.researchgate.net/profile/Eduardo_Velloso/publication/262162999_MotionMA_Motion_modelling_and_analysis_by_demonstration/links/577a4aaa08aece6c20fbc5bf.pdf
% - https://www.ncbi.nlm.nih.gov/pubmed/27555917
% - http://online.liebertpub.com/doi/abs/10.1089/g4h.2012.0041?url_ver=Z39.88-2003&rfr_id=ori%3Arid%3Acrossref.org&rfr_dat=cr_pub%3Dpubmed&
% - https://www.ncbi.nlm.nih.gov/pmc/articles/PMC4835340/
% - http://ieeexplore.ieee.org/document/7399879/

% - https://www.akqa.com/work/nike/kinect/
% - https://www.ea.com/de-de/news/ea-sports-active-2-bringt-fitnessfans-in-die-form-ihres-lebens

Feedback is the main and most powerful component of the SLS. Since the user should interact on her own with it one has to assume that no other person interferes with her and the system. With this in mind the feedback of the system should be designed in a way, that the user knows at any time what she has to do or has done (cf.~\textit{\hyperref[nielsenDesignPrinciples]{Aesthetic and minimalistic design}}). In general, audio and visual feedback will be provided to the user. Regarding the interaction with the system, e.g. clicking a button, the system should respond with an audio signal as well as changing the visual state of an element accordingly.

%For the accomplishment of the exercise execution it is essential to provide real-time feedback.
Real-time feedback supports the trainee during her performance and improves the learning effect when given in an appropriate manner~\cite{Hodges2002-gb, Liebermann2002-zr, Winstein1990-to}. The SLS should therefore respond to the user with immediate helpful information during the training to improve the exercise execution. This can be seen in several other sport applications like \textit{EA SPORTS Active 2}\footnote{\url{https://www.ea.com/de-de/news/ea-sports-active-2-bringt-fitnessfans-in-die-form-ihres-lebens}} or \textit{Nike +}\footnote{\url{https://news.nike.com/news/introducing-nike-kinect-training}}. The provided feedback should mainly indicate whether the current execution is performed correctly or not, visualize the the performance of the user regarding the predefined exercises, and show the execution progress. The user should also see herself mirrored in an appropriate environment to see her current execution and if she is in detection range of the tracking device. With this, a baseline is built for appropriate real time feedback.

\begin{comment}
- in general audio and visual feedback
- if the user has clicked a button
- stays in the right position
- Indicator about if exercise is correctly executed
- Indicator about how good the exercise is performed
- real time feedback (Time, confidence, checklist, repetitions)
- system should inform how many repetitions are left
- system should inform when the repetition is successfully accomplished (audio, visual -> timer green, repetition counter)
- system should inform the user if repetition was not successful (audio, visual -> reset timer, checklist)
- After successful execution, a summary is shown about the performance of the user for each rep (time, attempts, confidence)
- tier summary (avg. time, attempts, confidence)
\end{comment}
