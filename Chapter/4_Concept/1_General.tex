\section{Basic Principles}\label{4_1_general}
In general, the SLS should provide the following characteristics: easy to learn, understand, and a simple interaction technique.
Usability heuristics can be applied to identify and prevent interaction design problems to ensure an appropriate user experience.
%Appropriate user experience helps to achieve these .
%Usability heuristics are useful to identify or prevent problems in a system.
Therefore, the SLS will acknowledge the interaction design principles by Nielsen~\cite{Nielsen_1994-he} described in section \textit{\nameref{nielsenDesignPrinciples}}.
Beside that, certain tasks have to be considered that are more related to the system itself, like multiple user profiles or designing the exercise workflow.
An overview about these can be found in section \textit{\nameref{systemBasics}}.

\subsection{Ten Heuristic Principles for Interaction Design}\label{nielsenDesignPrinciples}
Nielsen designed his ten heuristics by comparing several sets of usability heuristics with existing usability problems from certain projects~\cite{Nielsen_1994-he}.
He was able to determine which heuristics identify usability problems the best and therefore created a set of them.
They can also be used as a guideline for designing and developing a user friendly system to prevent usability problems.
Examples of these can be found in the subsequent subsections.
The SLS will respect these interaction design principles described in the following.
\\\\
\textbf{\hyperref[4_1_1_visibilitySystemStatus]{Visibility of System Status}}\\
The system should always keep the user informed about the current state through appropriate feedback in an adequate time.\\
\textit{Example: Highlight if an exercise is executed correctly}\\

\textbf{Match between System and the Real World}\\
The system should provide the user with familiar terms and information. Using technical terms with which she is not familiar can lead to confusion. Therefore proper information should be natural and in a meaningful order.\\
\textit{Example: Usage of familiar terms in the description for the exercises}\\

\textbf{User Control and Freedom}\\
If the user clicks accidentally on something she should be able to leave this state without any troubles.\\
\textit{Example: Back button in each screen to return easily to the last screen}\\

\textbf{Consistency and Standards}\\
It should follow a clear design standard and provide consistency. The user should not be confused whether different terms or elements mean the same.\\
\textit{Example: Visualization of buttons are always the same in their context}\\

\textbf{Error Prevention}\\
Conditions and actions that could easily result in errors should be prevented. Another option is to inform the user about the consequences that the action may have and which she has to actively confirm.\\
\textit{Example: User should only start an exercise if she stands in the correct starting pose}\\

\textbf{Recognition rather than Recall}\\
The users memory load has to be minimized. She should not remember every action or information. Elements, actions, and options should be visible and instructions about the usage must be easy to retrieve.\\
\textit{Example: Clear instruction of exercises and providing hints about actions during the exercise execution}\\

\textbf{Flexibility and Efficiency of Use}\\
Providing quick options and allowing to skip certain steps can speed up the interaction for more familiar users. Hence the system should take care of both novice and experienced users.\\
\textit{Example: Skipping the tutorial of the system for experienced users}\\

\textbf{Aesthetic and Minimalist Design}\\
Information should just contain aspects that are relevant to the user and that she really needs. Every irrelevant data decreases the intelligibility.\\
\textit{Example: Providing relevant steps in the exercise description, time to hold the exercise, and the amount of repetitions}\\

\textbf{Help Users Recognize, Diagnose, and Recover from Errors}\\
Error messages should accurately indicate the ongoing problem such that the user knows what is wrong. Providing a constructive solution helps the user to solve the problem.\\
\textit{Example: Warn the user if she is not standing in the starting position to start the execution of the exercise}\\

\textbf{Help and Documentation}\\
Optimally the system can be used without any further documentation. If it cannot be circumvented the provided help and documentation should be easy to find and clearly show the relevant steps.\\
%If it cannot be circumvented it should be easy to find helpful documentation and clearly show the relevant steps.
\textit{Example: Provide a tutorial when interacting with the system for the first time}

\subsection{System Specific Basics}\label{systemBasics}
One person at a time should be able to interact with the SLS.
This is because mostly just one person can stay on the slackline especially for beginners.
However, it should provide the ability to have multiple user profiles.
Several people can thereby have their own profile in the same application.
For proper user training the system should follow a clear workflow.
Therefore two methods have been discussed in section \textit{\nameref{3_3_1_learningConcepts}}.
A methodical routine will be used with which levels and exercises can be designed.
These should be locked at the beginning and the user can unlock them by successfully executing the prior exercises.
Another important part is the user tracking.
The SLS should be able to track the user in an adequate accuracy and precision such that it can match the users' movement with the actual exercise. %The SLS should match the users' movement with the actual exercise in an adequate tracking accuracy and precision.
This is in correlation with properly providing real-time feedback, which is further discussed in section \textit{\nameref{4_5_feedbackSystem}}.
All relevant recorded data should be immediately saved when it is needed, e.g. when successfully accomplishing an exercise.

\begin{comment}
- System should be able to track user appropriately
- All relevant data should be immediately saved when it is needed (unlocking exercise/stage, failing/accomplish exercise)
- Information about where the user currently is should be given --> title
- User selection
- Also a possibility to go to the last screen if she misclicks should be given.
\end{comment}