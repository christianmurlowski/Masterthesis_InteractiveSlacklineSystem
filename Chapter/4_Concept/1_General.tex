\section{General Information}\label{4_2_general}
In general the system should follow the \textbf{\nameref{nielsenDesignPrinciples}} by Nielsen~\cite{Nielsen_1994-he}. With this a foundation of a good user experience can be built. Further some \textbf{\nameref{systemBasics}} have to be considered, that give an overview about certain task of the system.

\subsection{Ten heuristic principles for interaction design}\label{nielsenDesignPrinciples}
Nielsens heuristics help to identify usability problems. He created these by comparing several existing usability heuristics existing usability problems from a certain projects. With this he was able to determine what heuristics identify usability problems the best and creating this set of them. For preventing that the system even result in such problems they can be used as a guideline for designing and developing a user friendly system. In the following the set of these heuristics will be discussed.
\subsubsection{Visibility of system status}
The system should always keep user informed about the current state through appropriate feedback in an adequate time.
\subsubsection{Match beween system and the real world}
The system should provide the user with familiar terms and information. It should not confuse him with technical terms with which she is not familiar. Therefore information should be natural and in a meaningful order.
\subsubsection{User control and freedom}
If the user clicks accidently on something she should be able to leave this state without any troubles.
\subsubsection{Consistency and standards}
It should follow a clear design standard and therefore provide consistency. The user should not be distracted wether different terms or elements mean the same.
\subsubsection{Error prevention}
Conditions and actions that could easily result in errors should be prevented or the user should be informed about the consequences that the action may have and then has to actively commit it.
\subsubsection{Recognition rather than recall}
The users memory load should be restricted. She should not remember every action or information. Therefore elements, actions and options should be visible and instructions about the usage should be easily retrievable.
\subsubsection{Flexibility and efficiency of use}
Providing quick options and allowing to skip certain steps can speed up the interaction of more familiar users. Hence the system should take care of both novice and experienced users
\subsubsection{Aesthetic and minimalistic design}
Information should just contain aspects that are relevant to the user and that she really needs. Each irrelevant data decreases the intelligibility.
\subsubsection{Help users recognize, diagnose, and recover from errors}
Error messages should accurately indicate the ongoing problem such that the user knows what is wrong. Also it should provide a constructively solution.
\subsubsection{Help and documentation}
In the best case the system can be used without any further documentation. It may be the case to provide help and documentation. If so it should be easy to find it and clearly show the relevant steps.

\subsection{System specific basics}\label{systemBasics}
Since the system is autonomous it should provide the ability to have multiple user profiles, where they can switch between such that several persons can have a profile on the same application. Although just one person at a time should be able to interact with the system. This can also be seen in many others applications like \textbf{\todo{[CITE]}}. To train the user properly it should follow a clear workflow, where the \textbf{\nameref{3_2_2_StagesExercises}} are divided into levels that the user has to unlock. Another important part is the user tracking. The system should be able to track the user in an appropriate accuracy, such that it can match the users movement with the actual exercise. This is in correlation with properly providing real-time feedback, which is further discussed in \textbf{\nameref{4_5_feedbackSystem}}. Further all relevant data should be immediately saved when it is needed, for example when successfully accomplishing an exercise and so on.

\begin{comment}
- System should be able to track user appropriately
- All relevant data should be immediately saved when it is needed (unlocking exercise/stage, failing/accomplish exercise)
- Information about where the user currently is should be given --> title
- User selection
- Also a possibility to go to the last screen if she misclicks should be given.
\end{comment}