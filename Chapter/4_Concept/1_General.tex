\section{General Information}\label{4_1_general}
In general the system should be easy to understand, to learn, and to interact. To achieve this it should provide proper user experience. Usability heuristics are useful to identify or to prevent problems in a system. Therefore the concept will rely on the interaction design principles by Nielsen~\cite{Nielsen_1994-he} described in subsection \textit{\nameref{nielsenDesignPrinciples}}. Beside this certain tasks have to be considered that are more related to this system. An overview about these can be found in section \textit{\nameref{systemBasics}}.

\subsection{Ten heuristic principles for interaction design}\label{nielsenDesignPrinciples}
Nielsen designed his ten heuristics by comparing several sets of usability heuristics with existing usability problems from certain projects. Hence he was able to determine what heuristics identify usability problems the best and therefore creating a set of them. To prevent that a system results in having such problems they can also be used as a guideline for designing and developing a user friendly system. The interactive slackline system will follow these principles, which are described in the following:

\textbf{\hyperref[4_1_1_visibilitySystemStatus]{Visibility of system status}}\\
The system should always keep the user informed about the current state through appropriate feedback in an adequate time.\\

\textbf{Match between system and the real world}\\
The system should provide the user with familiar terms and information. Using technical terms with which she is not familiar can lead to confusion. Therefore proper information should be natural and in a meaningful order.\\

\textbf{User control and freedom}\\
If the user clicks accidently on something she should be able to leave this state without any troubles.\\

\textbf{Consistency and standards}\\
It should follow a clear design standard and provide consistency. The user should not be confused whether different terms or elements mean the same.\\

\textbf{Error prevention}\\
Conditions and actions that could easily result in errors should be prevented. Another option is to inform the user about the consequences that the action may have and which she has to actively confirm.\\

\textbf{Recognition rather than recall}\\
The users memory load has to be minimized. She should not remember every action or information. Elements, actions, and options should be visible and instructions about the usage must be easy to retrieve.\\

\textbf{Flexibility and efficiency of use}\\
Providing quick options and allowing to skip certain steps can speed up the interaction for more familiar users. Hence the system should take care of both novice and experienced users.\\

\textbf{Aesthetic and minimalistic design}\\
Information should just contain aspects that are relevant to the user and that she really needs. Every irrelevant data decreases the intelligibility.\\

\textbf{Help users recognize, diagnose, and recover from errors}\\
Error messages should accurately indicate the ongoing problem such that the user knows what is wrong. Providing a constructive solution helps the user to solve the problem.\\

\textbf{Help and documentation}\\
Optimally the system can be used without any further documentation. It may be the case to provide help and documentation. If it cannot be circumvented it should be easy to find it and clearly show the relevant steps.\\

\subsection{System specific basics}\label{systemBasics}
One person at a time should be able to interact with the system. This is because mostly just one person can stay on the slackline especially for beginners. However it should provide the ability to have multiple user profiles. One can switch between those such that several persons can have a profile on the same application. For proper user training the system should follow a clear workflow. Therefore two methods have been discussed in section \textit{\nameref{3_3_1_learningConcepts}}. A methodical routine will be used with which stages and exercises can be designed as levels. These should be locked at the beginning and the user can unlock them by successfully executing the prior exercise. Another important part is the user tracking. The system should be able to track the user in an appropriate accuracy such that it can match the users movement with the actual exercise. This is in correlation with properly providing real-time feedback, which is further discussed in section \textit{\nameref{4_5_feedbackSystem}}. All relevant recorded data should be immediately saved when it is needed, e.g. when successfully accomplishing an exercise.

\begin{comment}
- System should be able to track user appropriately
- All relevant data should be immediately saved when it is needed (unlocking exercise/stage, failing/accomplish exercise)
- Information about where the user currently is should be given --> title
- User selection
- Also a possibility to go to the last screen if she misclicks should be given.
\end{comment}