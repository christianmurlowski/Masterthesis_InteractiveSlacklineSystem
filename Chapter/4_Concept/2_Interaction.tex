\section{Interaction}\label{4_2_interaction}
The interaction can be seen as a bigger part of the system since it is independent of any external controlling devices. The user should be able to navigate through the system by herself with her hands as input for the interaction. A cursor should always be visualized to navigate through the systems interface. If the user initially starts the system, there should be an engagement gesture to convey that the system initially recognises and responds to a user action. Furthermore a small tutorial should be given in which the user will be trained on how to use the interaction possibilities with the system (\textit{cf. \hyperref[nielsenDesignPrinciples]{Recognition rather than recall}}). To make her familiar with these, she should directly apply these techniques in the tutorial. The current state of the interaction is clearly visualized, such that the user knows if she triggers an action regarding an element (\textit{cf .\hyperref[nielsenDesignPrinciples]{Visibility of system status}}). To be able to interact with elements and start the exercise execution the user should stay in a predefined initial position. The SLS then recognises if a user is ready to start. Interaction will also play a role in exercise execution. During the execution she interacts with the SLS by trying to match the predefined exercise. The user should then get appropriate feedback, which is further explained in section \textit{\nameref{4_5_feedbackSystem}}.

\begin{comment}
- user can and should interact with the system
\\- Cursor visualization as hand image
\\- Engagement gesture for first interaction with Kinect (One hand over shoulder)
\\- She should be instructed how to interact 
\\- Different interaction methods should be provided to prevent failing on one (tutorial --> clicking (variations) + scrolling)
\end{comment}