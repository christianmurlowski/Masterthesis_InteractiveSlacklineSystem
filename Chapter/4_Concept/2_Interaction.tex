\section{Interaction}\label{4_2_interaction}
Since the user can and should be able to navigate through the system by herself the interaction can be seen as one of the bigger parts of the system. The user can control a cursor with her own hand, which is also visualized in the screen. When initially starting the system the user should raise one hand over the head to convey that the system initially recognises and responds to a user action. 

Further a small tutorial is given in which the user is instructed in how to click by pushing his hand towards the Kinect or doing a point gesture and how to scroll through a menu. These actions have to be directly applied by the user. The state of the current interaction is visualized properly by providing a circle around the hand cursor that represents the progress like seen in figure \ref{fig:handcursorProgress}. Besides this she is instructed on how to stay in the right starting position. This is required by some actions like just before starting the exercise execution to ensure the user is ready.
\begin{figure}[htb]
	\centering
	\begin{minipage}[t]{1\linewidth}
		\centering
		\includegraphics[width=0.6\linewidth]{Pictures/handcursorProgress}
		\caption{Progress of handcursor (Left: Default, Middle: In progress, Right: Finished)}
		\label{fig:handcursorProgress}
	\end{minipage}
\end{figure}

Another big role of this plays in the exercise execution. Here the user interacts with the system to match a predefined gesture for accomplishing the exercise. Therefore real time feedback is provided which gives her hints about the right interaction and how good it is performed. More information about the feedback methods can be found in \textbf{\nameref{4_6_feedbackSystem}}.
\begin{comment}
- user can and should interact with the system
\\- Cursor visualization as hand image
\\- Engagement gesture for first interaction with Kinect (One hand over shoulder)
\\- She should be instructed how to interact 
\\- Different interaction methods should be provided to prevent failing on one (tutorial --> clicking (variations) + scrolling)
\end{comment}