\section{Interaction}\label{4_2_interaction}
The interaction can be seen as one of the bigger parts of the system since it is autonomous. So the user should be able to navigate through the system by herself with her hands as input for the interaction. A cursor should always be visualized to navigate through the systems interface. If the user initially starts the system, there should be an engagement gesture to convey that the system initially recognises and responds to a user action. Further a small tutorial should be given in which the user will be trained on how to use the interaction possibilities with the system. To make her familiar with these, she should directly apply these techniques in the tutorial. The current state of the interaction should be properly visualized, such that the user knows if she is in default mode or in progress of an interaction regarding an element. Sometimes to be able to interact with elements, the user should stay in the right position, so that the system knows if the user is ready to start. This could be useful before starting the actual exercise. Interaction will also play a role in exercise execution. In here the user interacts with the system by trying to match the predefined gestures. She should then get appropriate feedback, which is further explained in section \textit{\nameref{4_5_feedbackSystem}}.

\begin{comment}
- user can and should interact with the system
\\- Cursor visualization as hand image
\\- Engagement gesture for first interaction with Kinect (One hand over shoulder)
\\- She should be instructed how to interact 
\\- Different interaction methods should be provided to prevent failing on one (tutorial --> clicking (variations) + scrolling)
\end{comment}