\section{Stages}\label{4_3_stages}
The system contains predefined gestures, which are subdivided in stages, which can be seen in \textbf{\nameref{3_2_2_StagesExercises}}. A stage menu should be provided to give the user an overview and in which each one consists of several exercises. Every stage is locked except the first one, which is the users starting point. The next stage can be unlocked, by successfully executing all exercise in the current one. Hence it can be assured that the user is able to encounter with the more difficult exercises. The user should be introduced to the stage. In here the purpose, goal, and helpful techniques should be given, such that the user becomes an overview about the exercises. At last a summary scene shows several performance parameter for the exercises in this stage.
\begin{comment}
\\- system should provide predefined exercises that can be tracked per user
\\- A stage menu should be provided to the user, which shows her the amount of stages to complete
\\- It consists of 4 stages each with specific exercises (Preliminary, First contact with slacklining,  Static exercises, Dynamic exercises) like explained in chapter 3
\\- A stage consists of several exercises like explained in prev. chapter
\\- locked stages (initial first stage interactable)
\\- unlock stages (by successfully accomplishing all exercises from current stage)
\\- Stage introduction gives general information, describes the goal and tips for the current stage
-- a stage information scene provides her with the general introduction of this stage --> unlocks first exercise
\\- Stage summary contains average data about each exercise
\end{comment}
