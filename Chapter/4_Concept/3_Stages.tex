\section{Levels}\label{4_3_stages}
The SLS covers predefined exercises, which are subdivided in levels that have been discussed in section \textit{\nameref{3_3_2_StagesExercises}}. They follow a structure of a level design in which the user has to unlock each level to make progress. With this an exergame approach is followed to motivate the user for unlocking the next level. Therefore a menu should exist for all available stages as well as for all exercises within a stage. The very first stage and exercise should be unlocked and interactable to give the user a starting point. She can then unlock the next stage by accomplishing all exercises in the current one. In this way it can be ensured that the user is able to encounter with more difficult exercises in the next stage. An introduction into each stage should inform the user about the purpose and goals of it, as well as general information about the exercise within that stage. Lastly a summary gives an overview of her performance for the entire stage.

%The user should also be introduce in each exercise to know how to perform it correctly and give her support for successfully executing it
