\section{Stages}\label{4_3_stages}
The system contains predefined gestures, which are subdivided in stages that have been elaborated in \textbf{\nameref{3_3_2_StagesExercises}}. Since the interactive slackline system follows a slightly exergame like approach, the stages and exercises should be designed as levels, which the user could select has to unlock. Therefore a menu should exist for all available stages as well as for all exercises within a stage. To give her a starting position, the very first stage and exercise should be interactable. 
She can then unlock the next stage by accomplishing all exercises in the last one. Hence it can be assured that the user is able to encounter with the more difficult exercises. She should also be introduce in each stage to know how its purpose and goal. At last a summary can be given to show an overview of her performance for the entire stage.

%The user should also be introduce in each exercise to know how to perform it correctly and give her support for successfully executing it
