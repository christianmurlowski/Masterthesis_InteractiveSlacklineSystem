\section{Scenario}\label{4_6_scenario}
To have a better understanding regarding the interplay of the several components a generic scenario workflow will be given from a users' point of view. The users' name is Bob and he is 21 years old. While climbing with friends in a climbing hall he noticed a new interactive learning system that they provide to visitors for trying and learning slacklining. He heard once of this activity by friends but never had the chance to try it. So he decides to test it and wants to execute some exercises. To engage with the system he has to stay in front of it, which recognizes him immediately. At the beginning it introduces him regarding the interaction possibilities. After that, he selects a tier and is further informed about the goal and basics of this stage. Once confirming that he read everything it leads him to the first available exercise. The system shows him how to execute it properly. Right after starting the exercise he gets helpful real time feedback to correct himself for a successful accomplishment of the execution. After finishing the exercise Bob gets an overview about his performance.

%asks him to do an engagement gesture. Now he is introduced in the interaction of the system. After he knows how to interact with the system he can select a user profile, which was created by the personal before. This will lead him to the stage selection. He selects the very first because all others are locked. In here bob has to read the instructions for the stage to be informed about what he has to pay attention for. Next he selects the first exercise and a body side he wants to train. Before bob can start with the execution, he will be informed about how this specific exercise is executed and how to perform it. After he thinks that he is ready to challenge it, he starts the exercise explicitly. Now he sees all relevant information about the execution and starts to execute it. He notices that the system gives him real time feedback about his performance and the progress made. After finishing all repetition the system leads him to the exercise summary screen, which gives an overview about the performance of the just executed exercise. 

%The user raises the hands over her head to start the application. She is now instructed on how to interact with elements on  the screen, e.g. clicking and scrolling. After being confident with this, she selects her user profile to load the appropriate exercises and leads her to the stage selection. She selects the first stage since all others are currently locked. Now she is in the exercise menu. In here she clicks on the \textit{stage information} button, which gives her an introduction into the stage. After confirming that she has read the introduction the first exercise becomes unlocked and she selects it. After that she decides to train her left leg first in the side selection. An exercise introduction screen is following, which shows specific information about the execution. After reading the introduction she feels ready to counter the exercise. Therefore she goes into the starting position and starts the exercise by clicking the start button.

%The screen changes and all relevant elements are shown for the exercise execution. She performs every repetition of the exercise successfully. After finishing these the system leads her to the exercise summary screen, which shows her the performance of the just executed exercise. In here she can now decide to return to the main menu or go on and start the training for the other body side. This procedure is also visualized in Figure \ref{fig:scenarioWorkflow} below.
