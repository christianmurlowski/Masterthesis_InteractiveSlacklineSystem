\chapter{Concept}\label{4_concept}
This chapter describes the conceptual analysis of an interactive slackline learning system (SLS) with real-time feedback. The idea of the SLS is to provide helpful information, structured exercises, and appropriate feedback to the user for learning slacklining with the given application. 
%One main feature is the autonomous interaction with the system. 
The user should be able to interact independently of any controlling device or external support like human help. Therefore, the SLS can only be controlled by the currently interacting user. Further, it responds appropriately to the actions of the user and provides several real-time feedback indicators to support her during the exercise execution. 

In the following conceptual analysis will be elaborated. Section \textit{\nameref{4_1_general}} describes basic design principles and system related requirements. This is followed by the more specific sections \textit{\nameref{4_2_interaction}}, \textit{\nameref{4_3_stages}}, and \textit{\nameref{4_4_exercises}} that describe how to interact with the system and how exercises are structured. Another main component is to provide adequate feedback to the user, which is part of section \textit{\nameref{4_5_feedbackSystem}}. Lastly section \textit{\nameref{4_6_scenario}} gives a good overview about the workflow of the specific components.
\section{Basic Principles}\label{4_1_general}
In general, the SLS should provide the following characteristics: easy to learn, understand, and a simple interaction technique.
Usability heuristics can be applied to identify and prevent interaction design problems to ensure an appropriate user experience.
%Appropriate user experience helps to achieve these .
%Usability heuristics are useful to identify or prevent problems in a system.
Therefore, the SLS will acknowledge the interaction design principles by Nielsen~\cite{Nielsen_1994-he} described in section \textit{\nameref{nielsenDesignPrinciples}}.
Beside that, certain tasks have to be considered that are more related to the system itself, like multiple user profiles or designing the exercise workflow.
An overview about these can be found in section \textit{\nameref{systemBasics}}.

\subsection{Ten Heuristic Principles for Interaction Design}\label{nielsenDesignPrinciples}
Nielsen designed his ten heuristics by comparing several sets of usability heuristics with existing usability problems from certain projects~\cite{Nielsen_1994-he}.
He was able to determine which heuristics identify usability problems the best and therefore created a set of them.
They can also be used as a guideline for designing and developing a user friendly system to prevent usability problems.
Examples of these can be found in the subsequent subsections.
The SLS will respect these interaction design principles described in the following.
\\\\
\textbf{\hyperref[4_1_1_visibilitySystemStatus]{Visibility of System Status}}\\
The system should always keep the user informed about the current state through appropriate feedback in an adequate time.\\
\textit{Example: Highlight if an exercise is executed correctly}\\

\textbf{Match between System and the Real World}\\
The system should provide the user with familiar terms and information. Using technical terms with which she is not familiar can lead to confusion. Therefore proper information should be natural and in a meaningful order.\\
\textit{Example: Usage of familiar terms in the description for the exercises}\\

\textbf{User Control and Freedom}\\
If the user clicks accidentally on something she should be able to leave this state without any troubles.\\
\textit{Example: Back button in each screen to return easily to the last screen}\\

\textbf{Consistency and Standards}\\
It should follow a clear design standard and provide consistency. The user should not be confused whether different terms or elements mean the same.\\
\textit{Example: Visualization of buttons are always the same in their context}\\

\textbf{Error Prevention}\\
Conditions and actions that could easily result in errors should be prevented. Another option is to inform the user about the consequences that the action may have and which she has to actively confirm.\\
\textit{Example: User should only start an exercise if she stands in the correct starting pose}\\

\textbf{Recognition rather than Recall}\\
The users memory load has to be minimized. She should not remember every action or information. Elements, actions, and options should be visible and instructions about the usage must be easy to retrieve.\\
\textit{Example: Clear instruction of exercises and providing hints about actions during the exercise execution}\\

\textbf{Flexibility and Efficiency of Use}\\
Providing quick options and allowing to skip certain steps can speed up the interaction for more familiar users. Hence the system should take care of both novice and experienced users.\\
\textit{Example: Skipping the tutorial of the system for experienced users}\\

\textbf{Aesthetic and Minimalist Design}\\
Information should just contain aspects that are relevant to the user and that she really needs. Every irrelevant data decreases the intelligibility.\\
\textit{Example: Providing relevant steps in the exercise description, time to hold the exercise, and the amount of repetitions}\\

\textbf{Help Users Recognize, Diagnose, and Recover from Errors}\\
Error messages should accurately indicate the ongoing problem such that the user knows what is wrong. Providing a constructive solution helps the user to solve the problem.\\
\textit{Example: Warn the user if she is not standing in the starting position to start the execution of the exercise}\\

\textbf{Help and Documentation}\\
Optimally the system can be used without any further documentation. If it cannot be circumvented the provided help and documentation should be easy to find and clearly show the relevant steps.\\
%If it cannot be circumvented it should be easy to find helpful documentation and clearly show the relevant steps.
\textit{Example: Provide a tutorial when interacting with the system for the first time}

\subsection{System Specific Basics}\label{systemBasics}
One person at a time should be able to interact with the SLS.
This is because mostly just one person can stay on the slackline especially for beginners.
However, it should provide the ability to have multiple user profiles.
Several people can thereby have their own profile in the same application.
For proper user training the system should follow a clear workflow.
Therefore two methods have been discussed in section \textit{\nameref{3_3_1_learningConcepts}}.
A methodical routine will be used with which levels and exercises can be designed.
These should be locked at the beginning and the user can unlock them by successfully executing the prior exercises.
Another important part is the user tracking.
The SLS should be able to track the user in an adequate accuracy and precision such that it can match the users' movement with the actual exercise. %The SLS should match the users' movement with the actual exercise in an adequate tracking accuracy and precision.
This is in correlation with properly providing real-time feedback, which is further discussed in section \textit{\nameref{4_5_feedbackSystem}}.
All relevant recorded data should be immediately saved when it is needed, e.g. when successfully accomplishing an exercise.

\begin{comment}
- System should be able to track user appropriately
- All relevant data should be immediately saved when it is needed (unlocking exercise/stage, failing/accomplish exercise)
- Information about where the user currently is should be given --> title
- User selection
- Also a possibility to go to the last screen if she misclicks should be given.
\end{comment}
\section{Interaction}\label{4_2_interaction}
Since the user can and should be able to navigate through the system by herself the interaction can be seen as one of the bigger parts of the system. The user can control a cursor with her own hand, which is also visualized in the screen. When initially starting the system the user should raise one hand over the head to convey that the system initially recognises and responds to a user action. 

Further a small tutorial is given in which the user is instructed in how to click by pushing his hand towards the Kinect or doing a point gesture and how to scroll through a menu. These actions have to be directly applied by the user. The state of the current interaction is visualized properly by providing a circle around the hand cursor that represents the progress like seen in figure \ref{fig:handcursorProgress}. Besides this she is instructed on how to stay in the right starting position. This is required by some actions like just before starting the exercise execution to ensure the user is ready.
\begin{figure}[htb]
	\centering
	\begin{minipage}[t]{1\linewidth}
		\centering
		\includegraphics[width=0.6\linewidth]{Pictures/handcursorProgress}
		\caption{Progress of handcursor (Left: Default, Middle: In progress, Right: Finished)}
		\label{fig:handcursorProgress}
	\end{minipage}
\end{figure}

Another big role of this plays in the exercise execution. Here the user interacts with the system to match a predefined gesture for accomplishing the exercise. Therefore real time feedback is provided which gives her hints about the right interaction and how good it is performed. More information about the feedback methods can be found in \textbf{\nameref{4_6_feedbackSystem}}.
\begin{comment}
- user can and should interact with the system
\\- Cursor visualization as hand image
\\- Engagement gesture for first interaction with Kinect (One hand over shoulder)
\\- She should be instructed how to interact 
\\- Different interaction methods should be provided to prevent failing on one (tutorial --> clicking (variations) + scrolling)
\end{comment}
\section{Stages}\label{4_3_stages}
The system contains predefined gestures, which are subdivided in stages that have been elaborated in \textbf{\nameref{3_3_2_StagesExercises}}. Since the interactive slackline system follows a slightly exergame like approach, the stages and exercises should be designed as levels, which the user could select has to unlock. Therefore a menu should exist for all available stages as well as for all exercises within a stage. To give her a starting position, the very first stage and exercise should be interactable. 
She can then unlock the next stage by accomplishing all exercises in the last one. Hence it can be assured that the user is able to encounter with the more difficult exercises. She should also be introduce in each stage to know how its purpose and goal. At last a summary can be given to show an overview of her performance for the entire stage.

%The user should also be introduce in each exercise to know how to perform it correctly and give her support for successfully executing it

\section{Exercises}\label{4_4_exercises}
%should have predefined exercises that can be tracked in an appropriate manner
Each exercise is part of one stage. An exercise itself is divided into two body sides, which are further divided into several repetitions, see figure \ref{fig:exerciseStructure}. Each exercise is locked except the first one to provide a starting point. The next exercise can be unlocked by accomplishing both sides of the current exercise. Similarly a side is completed if all repetitions have been finished. Like for the stage, each exercise should be instructed for the user, such that she can successfully perform it. She should stand in a starting position to start the exercise. This is to ensure that no exercise is starting to track if the user would make a random gesture which could lead to confusion of the user. During the execution she should get real time feedback about her current performance, which is further discussed in \textbf{\nameref{4_6_feedbackSystem}}. After the execution a summary should show the performance of the execution with several parameters like execution time, amount of attempts needed, and the confidence regarding the given gesture.

\begin{figure}[htb]
	\centering
	\begin{minipage}[t]{1\linewidth}
		\centering
		\includegraphics[width=1\linewidth]{Pictures/exerciseStructureTopDown2}
		\caption{Exercise structure}
		\label{fig:exerciseStructure}
	\end{minipage}
\end{figure}
\begin{comment}
\\\textbf{Exercises}
\\- System should provide predefined exercises that user can execute and her performance is tracked
\\- main menu / exercise selection
\\- lock / unlock 
\\- Each next exercise is unlocked if current exercise is successfully completed
\\- Each exercise counts as accomplished if both sides has been successfully trained
\\- click on exercise -> side selection - Each exercise consists of 2 sides, for train left and right side
\\- per exercise there should be an introduction on how to perform it successfully (instruction tip list, amount of repetitions, minimum time to hold a gesture, looping video)
\\- user has to stay in a starting position (both legs parallel to each other)
\\- exercise execution (user mirror)
\end{comment}

\section{Feedback system}\label{4_5_feedbackSystem}
%The user should get feedback about her performance during exercise execution. 
%Another big role of this plays in the exercise execution. Here the user interacts with the system to match a predefined gesture for accomplishing the exercise. Therefore real time feedback is provided which gives her hints about the right interaction and how good it is performed. More information about the feedback methods can be found in feedback system

% - https://www.researchgate.net/profile/Eduardo_Velloso/publication/262162999_MotionMA_Motion_modelling_and_analysis_by_demonstration/links/577a4aaa08aece6c20fbc5bf.pdf
% - https://www.ncbi.nlm.nih.gov/pubmed/27555917
% - http://online.liebertpub.com/doi/abs/10.1089/g4h.2012.0041?url_ver=Z39.88-2003&rfr_id=ori%3Arid%3Acrossref.org&rfr_dat=cr_pub%3Dpubmed&
% - https://www.ncbi.nlm.nih.gov/pmc/articles/PMC4835340/
% - http://ieeexplore.ieee.org/document/7399879/

% - https://www.akqa.com/work/nike/kinect/
% - https://www.ea.com/de-de/news/ea-sports-active-2-bringt-fitnessfans-in-die-form-ihres-lebens

Feedback is the main and most powerful component of the interactive learning system. Since the user should interact on her own with the system one has to assume that no other person interfere with her and the system. With this in mind the feedback of the system should designed in a way, that the user knows at any time what she has to do or has done (cf.~\textit{\hyperref[nielsenDesignPrinciples]{Aesthetic and minimalistic design}}). In general audio and visual feedback will be provided to the user. Regarding the interaction with the system, e.g. clicking a button, the system should respond with an audio signal as well as change the elements visual state accordingly.

%For the accomplishment of the exercise execution it is essential to provide real-time feedback. 
Real-time feedback supports the trainee during her performance and enhances the learning effect more than with post feedback methods (cf.~\textit{\nameref{2_3_3_feedbackApproachesTechniques}}). Hence the overall exercise execution as well as its accomplishment can be improved by providing appropriate real-time feedback. The system should therefore response to the user with helpful information. This can be seen in several other sport applications like \textit{EA SPORTS Active 2}\footnote{\url{https://www.ea.com/de-de/news/ea-sports-active-2-bringt-fitnessfans-in-die-form-ihres-lebens}} or \textit{Nike +}\footnote{\url{https://news.nike.com/news/introducing-nike-kinect-training}}. The provided feedback should mainly indicate if the current execution is performed correctly or not, visualize the the performance of the user regarding the predefined exercises, and show the execution progress. The user should also see herself mirrored in an appropriate environment to see her current execution and if she is in detection range of the tracking device. With this a baseline is built for appropriate real time feedback.

\begin{comment}
- in general audio and visual feedback
- if the user has clicked a button
- stays in the right position
- Indicator about if exercise is correctly executed
- Indicator about how good the exercise is performed
- real time feedback (Time, confidence, checklist, repetitions)
- system should inform how many repetitions are left
- system should inform when the repetition is successfully accomplished (audio, visual -> timer green, repetition counter)
- system should inform the user if repetition was not successful (audio, visual -> reset timer, checklist)
- After successful execution, a summary is shown about the performance of the user for each rep (time, attempts, confidence)
- tier summary (avg. time, attempts, confidence)
\end{comment}

\section{Scenario}\label{4_6_scenario}
To have a better understanding regarding the interplay of the several components a generic scenario workflow will be given from the point of view of the user:

The user raises the hands over her head to start the application. She is now instructed on how to interact with elements on  the screen, e.g. clicking and scrolling. After being confident with this, she selects her user profile to load the appropriate exercises and leads her to the stage selection. She selects the first stage since all others are currently locked. Now she is in the exercise menu. In here she clicks on the \textit{stage information} button, which gives her an introduction into the stage. After confirming that she has read the introduction the first exercise becomes unlocked and she selects it. After that she decides to train her left leg first in the side selection. An exercise introduction screen is following, which shows specific information about the execution. After reading the introduction she feels ready to counter the exercise. Therefore she goes into the starting position and starts the exercise by clicking the start button. The screen changes and all relevant elements are shown for the exercise execution. She performs every repetition of the exercise successfully. After finishing these the system leads her to the exercise summary screen, which shows her the performance of the just executed exercise. In here she can now decide to return to the main menu or go on and start the training for the other body side. This procedure is also visualized in Figure \ref{fig:scenarioWorkflow} below.
\begin{comment}
\todo{-maybe just user view scenario}
To have a better understanding regarding the interplay of the several components a generic scenario workflow will be given.
The user starts with an engagement gesture like raising her hand over the head to convey that the system initially recognises and responds to a user action. After that a tutorial about the interaction with the system will be given that covers clicking and scrolling techniques. Now she's confident with the system interaction and can select a profile in the user select to train. This loads the profile which leads to the stage selection menu. In here she can select a stage, whereas initially the first one is can be selected and the others have to be unlocked by successfully accomplishing all exercises in the preview stage. Selecting a stage leads to the exercise menu. In here she has to read initially the stage introduction to become a basic understanding about the exercises in here. After reading this, it unlocks the first exercise. Selecting an exercise leads to the side selection, where the user has to choose the side she wants to train for this exercise. This is followed by an introduction of the exercise, in which is explained how to perform it correctly. If the user is ready, she should stay in a starting position to be able to start the exercise execution. In here she find all relevant elements to perform the exercise, like indicators for the time, repetitions, confidence and a checklist, which helps her to correctly execute the exercise. After successfully executing the exercise, a summary is shown which summarizes the user performance. Then she can return to the main menu or directly approach the next exercise. A stage summary gives an overview about all exercises with average performance parameters.
\end{comment}

\begin{figure}[htb]
	\centering
	\begin{minipage}[t]{1\linewidth}
		\centering
		\includegraphics[width=1\linewidth]{Pictures/conceptScenarioFlow2}
		\caption{Scenario workflow}
		\label{fig:scenarioWorkflow}
	\end{minipage}
\end{figure}
\section{Conclusion}\label{4_7_conclusion}