\section*{Abstract}
% The abstract should state
% (1) the principal objectives and scope of the investigation,
% (2) describe the methods employed,
% (3) summarize the results, and
% (4) state the principal conclusions.
\begin{comment}
- Einleitung
- Problem/Ziele
- Vorgesteller lösungsansatz
-- Exergame
-- SLS
-- Real time feedback
- einbau methodischer reihe als lernmethodik
- vergleich zu üblichen personal trainer
- Messung
- Results
- Conclusion


- trend von interaktiven lernmethodiken in allen bereichen des sports
- nichts vergeichbares in slacklinen
- Führen des nutzers mithilfe einer bestimmten lernmethodik und routine
- studie
\end{comment}
% Introduction
Combining sport activities with interactive technology is an ongoing and growing trend.
With the help of interactive devices, like the Kinect for Windows, applications can be made that guide the user through exercises to keep herself fit or for learning new sports.
Slacklining is a balancing sport on which the user has to walk over a narrow ribbon. The common methods of learning this sport are repetitive trials, which is not very effective, or having an intermediate slacker as a personal human trainer.
% Problems
However, this comes with several constraints. The beginner needs to know someone that has fundamental knowledge about slacklining. Further, she is dependent on him and his knowledge. A trainer cannot give constant real time feedback about the performance of the slacker.
% Solution approach
An interactive slackline training system is a first approach provides an intelligent training system, independency of any other help, and constant real time feedback. Further it can be used to store the progress of specific exercises and train several persons with the same system.
% Methods
In a user study it has been compared against the personal trainer with beginners that had no prior experience with slacklining or general balance activities.

%Results
\todo{- results}

%Conclusion
\todo{- conclusion}

%The aim of this thesis project is to design and evaluate an interactive slackline learning training system. To evaluate the efficiency of the system the learning progress is measured while training beginners on a slackline.

%In the study all participants have to execute predefined exercises that are divided into four training levels. Each exercise attempt, the time, and their confidence will be measured. Exercises will be executed without as well as with the slackline. For safety reasons gymnastic pads are placed beneath and to each side of the slackline.

%During the actual exercise study, the participant will be recorded via audio and video. This will only be used for this study to evaluate and validate performance parameters and detect failures of the system. The participant will be interviewed about her experience after the training. During the interview the participant will be recorded via audio for later analyses.

\begin{comment}
\begin{itemize}
\item Kinect v2
\item Slackline
\item No interactive system
\item Just learning by doing
\item No real-time feedback to clarify if something is wrong
\item System gives this feedback
\item Measure learn progress
\item In here user will be guided through exercises
\item If such an interactive system can be used to learn user to go on a slackline
\item If such a system is at least comparable with a human trainer (maybe also video training) or even better
\item User study with beginners in 2 groups
\item Problem
\item Current approaches to solve the problem
\item Problems with solutions, what would be better, what can't you do with these
\item Solution approach of thesis to make everything better, What will be better
\item Developed prototype, Used techniques, prototype design to motivate user
\item Pre-study / Feasibility study --> Why doing this? What evaluate?
\item Solve occurring problems, which exercises can be implemented + good trackable
\item Main study --> what will be tested?, number of exercises and tiers/stages and exercise levels 
\item Final results, What was good, What was bad, Usability of system
\item Assumed hypothesis true or false
\end{itemize}
\end{comment}
